\documentclass[12pt]{article}

\usepackage{url}

\title{Dieci consigli per una buona presentazione scientifica}
\author{Tu}

\begin{document}
\maketitle

\section{Introduzione}

Il testo di questo esercizio \`e una versione tradotta e significativamente riassunta di un eccellente articolo di Mark Schoeberl e Brian Toon dal titolo originale "Ten Secrets to Giving a Good Scientific Talk": 
\url{http://www.cgd.ucar.edu/cms/agu/scientific_talk.html}

\section{I consigli}

Ho compilato questa lista di ``consigli'' dopo aver ascoltato molti bravi presentatori e moltissimi inefficaci. Non pretendo che la lista sia onnicomprensiva - sono certo di aver dimenticato qualcosa. Tuttavia, la mia lista probabilmente copre il 90\% di quello che dovreste saper fare.

\begin{enumerate}

\item Organizza il tuo materiale attentamente, e con logica. Racconta una storia.

\item Fai pratica con la presentazione. Non ci sono scuse per la mancanza di preparazione.

\item Non inserire troppo materiale. Un buon speaker avr\`a uno o due punti principali, e parler\`a solo di quelli.

\item Evita le formule. Si dice che per ogni formula in una presentazione, il numero di persone in grado di comprenderla viene dimezzato. In altre parole, se $q$ è il numero di formule in una presentazione, ed $n$ il numero di persone che la possono comprendere, si ha che:

\begin{equation}
n = \gamma \left( \frac{1}{2} \right)^q
\end{equation}

dove $\gamma$ è una costante di proporzionalit\`a.

\item Concludi con pochi punti. La maggior parte delle persone non riuscir\`a a ricordare pi\`u di un paio di cose di una presentazione, specialmente se ne ascoltano molte in un breve intervallo di tempo.

\item Parla al pubblico, e non allo schermo. Uno degli errori pi\`u comuni che vedo, è quello di uno speaker che non stacca gli occhi dal monitor del computer.

\item Evita di emettere suoni che possano distrarre il pubblico. Tenta di evitare ``Mhhhh'' o ``Ehhhh'' tra una frase e l'altra.

\item Usa una grafica pulita. Ecco una serie di suggerimenti per le immagini:

\begin{itemize}

\item Usa lettere grandi.
\item Usa grafici semplici. Non mostrare grafici inutili.
\item Usa i colori.

\end{itemize}

\item Sii gentile ed esaustivo nel rispondere alle domande.

\item Usa un po' di umorismo se possibile. Mi stupisco sempre di come una battuta, per quando modesta, spezzi la tensione durante una presentazione.

\end{enumerate}

\end{document}
