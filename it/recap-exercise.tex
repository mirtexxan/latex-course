\documentclass[12pt]{article}

\usepackage{url}

\begin{document}
Dieci consigli per una buona presentazione scientifica

Scritto da: Te

Introduzione

Il testo di questo esercizio è una versione tradotta e significativamente riassunta di un eccellente articolo di Mark Schoeberl e Brian Toon dal titolo originale "Ten Secrets to Giving a Good Scientific Talk": 
\url{http://www.cgd.ucar.edu/cms/agu/scientific_talk.html}

I consigli

Ho compilato questa lista di "consigli" dopo aver ascoltato molti bravi presentatori e moltissimi inefficaci. Non pretendo che la lista sia onnicomprensiva - sono certo di aver dimenticato qualcosa. Tuttavia, la mia lista probabilmente copre il 90% di quello che dovreste saper fare.

1) Organizza il tuo materiale attentamente, e con logica. Racconta una storia.

2) Fai pratica con la presentazione. Non ci sono scuse per la mancanza di preparazione.

3) Non inserire troppo materiale. Un buon speaker avrà uno o due punti principali, e parlerà solo di quelli.

4) Evita le formule. Si dice che per ogni formula in una presentazione, il numero di persone in grado di comprenderla viene dimezzato. In altre parole, se q è il numero di formule in una presentazione, ed n il numero di persone che la possono comprendere, si ha che:

n = gamma per (1/2) alla q-esima potenza

dove gamma è una costante di proporzionalità.

5) Concludi con pochi punti. La maggior parte delle persone non riuscirà a ricordare più di un paio di cose di una presentazione, specialmente se ne ascoltano molte in un breve intervallo di tempo.

6) Parla al pubblico, e non allo schermo. Uno degli errori più comuni che vedo, è quello di uno speaker che non stacca gli occhi dal monitor del computer.

7) Evita di emettere suoni che possano distrarre il pubblico. Tenta di evitare "Mhhhh" o "Ehhhh" tra una frase e l'altra.

8) Usa una grafica pulita. Ecco una serie di suggerimenti per le immagini:

* Usa lettere grandi.

* Usa grafici semplici. Non mostrare grafici inutili.

* Usa i colori.

9) Sii gentile ed esaustivo nel rispondere alle domande.

10) Usa un po' di umorismo se possibile. Mi stupisco sempre di come una battuta, per quando modesta, spezzi la tensione durante una presentazione.

\end{document}
