\documentclass{beamer}

\input{preamble.tex}

\subtitle{Parte 2: Verso documenti strutturati \& oltre}

\begin{document}

%%%%%%%%%%%%%%%%%%%%%%%%%%%%%%%%%%%%%%%%%%%%%%%%%%%%%%%%%%%%%%%%%%%%%%%%%%%%%%%
%%%%%%%%%%%%%%%%%%%%%%%%%%%%%%%%%%%%%%%%%%%%%%%%%%%%%%%%%%%%%%%%%%%%%%%%%%%%%%%
%%%%%%%%%%%%%%%%%%%%%%%%%%%%%%%%%%%%%%%%%%%%%%%%%%%%%%%%%%%%%%%%%%%%%%%%%%%%%%%
\begin{frame}
\titlepage
\end{frame}

%%%%%%%%%%%%%%%%%%%%%%%%%%%%%%%%%%%%%%%%%%%%%%%%%%%%%%%%%%%%%%%%%%%%%%%%%%%%%%%
%%%%%%%%%%%%%%%%%%%%%%%%%%%%%%%%%%%%%%%%%%%%%%%%%%%%%%%%%%%%%%%%%%%%%%%%%%%%%%%
%%%%%%%%%%%%%%%%%%%%%%%%%%%%%%%%%%%%%%%%%%%%%%%%%%%%%%%%%%%%%%%%%%%%%%%%%%%%%%%
\section{Documenti strutturati}

%%%%%%%%%%%%%%%%%%%%%%%%%%%%%%%%%%%%%%%%%%%%%%%%%%%%%%%%%%%%%%%%%%%%%%%%%%%%%%%
%%%%%%%%%%%%%%%%%%%%%%%%%%%%%%%%%%%%%%%%%%%%%%%%%%%%%%%%%%%%%%%%%%%%%%%%%%%%%%%
%%%%%%%%%%%%%%%%%%%%%%%%%%%%%%%%%%%%%%%%%%%%%%%%%%%%%%%%%%%%%%%%%%%%%%%%%%%%%%%
\begin{frame}{Indice}
\begin{multicols}{2}
\tableofcontents[currentsection]
\end{multicols}
\end{frame}

%%%%%%%%%%%%%%%%%%%%%%%%%%%%%%%%%%%%%%%%%%%%%%%%%%%%%%%%%%%%%%%%%%%%%%%%%%%%%%%
%%%%%%%%%%%%%%%%%%%%%%%%%%%%%%%%%%%%%%%%%%%%%%%%%%%%%%%%%%%%%%%%%%%%%%%%%%%%%%%
%%%%%%%%%%%%%%%%%%%%%%%%%%%%%%%%%%%%%%%%%%%%%%%%%%%%%%%%%%%%%%%%%%%%%%%%%%%%%%%
\begin{frame}{\insertsection}
\begin{itemize}
\item Nella Parte 1, abbiamo imparato i comandi e gli ambienti di base per
la \structure{composizione} del testo.
\item In questa parte, impareremo i comandi e gli ambienti di base per
la \structure{strutturazione} del resto
\item Potete provare i nuovi comandi con \wllogo:
\end{itemize}
\vskip 2em
\begin{center}
\fbox{\href{\wlnewdoc{basics.tex}}{%
Clicca qui per aprire il documento di esempio in \wllogo{}}}
\\[1ex]\scriptsize{}
Per migliore compatibilit\`a, usate \href{http://www.google.com/chrome}{Chrome} o un  \href{http://www.mozilla.org/en-US/firefox/new/}{FireFox} recente.
\end{center}
\vskip 2ex
\begin{itemize}
\item E adesso\ldots iniziamo!
\end{itemize}
\end{frame}

%%%%%%%%%%%%%%%%%%%%%%%%%%%%%%%%%%%%%%%%%%%%%%%%%%%%%%%%%%%%%%%%%%%%%%%%%%%%%%%
%%%%%%%%%%%%%%%%%%%%%%%%%%%%%%%%%%%%%%%%%%%%%%%%%%%%%%%%%%%%%%%%%%%%%%%%%%%%%%%
%%%%%%%%%%%%%%%%%%%%%%%%%%%%%%%%%%%%%%%%%%%%%%%%%%%%%%%%%%%%%%%%%%%%%%%%%%%%%%%
\subsection{Titolo e sommario}
\begin{frame}[fragile]{\insertsubsection}
\begin{itemize}{\small
\item Comunicate a \LaTeX{} titolo \cmdbs{title} e autore \cmdbs{author} nel preambolo.
\item Usate \cmdbs{maketitle} nel corpo per comporre il titolo.
\item Usate l'ambiente \bftt{abstract} per creare un sommario.
\item Per ottenere i nomi degli elementi in italiano, si usa \cmdbs{babel}
}\end{itemize}
\begin{minipage}{0.55\linewidth}
\inputminted[fontsize=\scriptsize,frame=single,resetmargins]{latex}%
  {structure-title.tex}
\end{minipage}
\begin{minipage}{0.35\linewidth}
\includegraphics[width=\textwidth,clip,trim=2.2in 7in 2.2in 2in]{structure-title.pdf}
\end{minipage}
\end{frame}

%%%%%%%%%%%%%%%%%%%%%%%%%%%%%%%%%%%%%%%%%%%%%%%%%%%%%%%%%%%%%%%%%%%%%%%%%%%%%%%
%%%%%%%%%%%%%%%%%%%%%%%%%%%%%%%%%%%%%%%%%%%%%%%%%%%%%%%%%%%%%%%%%%%%%%%%%%%%%%%
%%%%%%%%%%%%%%%%%%%%%%%%%%%%%%%%%%%%%%%%%%%%%%%%%%%%%%%%%%%%%%%%%%%%%%%%%%%%%%%
\subsection{Sezioni}
\begin{frame}{\insertsubsection}
\begin{itemize}{\small
\item Dividere il documento in sezioni e sottosezioni \`e semplice:
basta usare \cmdbs{section} e \cmdbs{subsection}.
\item Riuscite ad indovinare cosa fanno \cmdbs{section*} e \cmdbs{subsection*}?
}\end{itemize}
\begin{minipage}{0.55\linewidth}
\inputminted[fontsize=\scriptsize,frame=single,resetmargins]{latex}%
  {structure-sections.tex}
\end{minipage}
\begin{minipage}{0.35\linewidth}
\includegraphics[width=\textwidth,clip,trim=1.5in 6in 4in 1in]{structure-sections.pdf}
\end{minipage}
\end{frame}

%%%%%%%%%%%%%%%%%%%%%%%%%%%%%%%%%%%%%%%%%%%%%%%%%%%%%%%%%%%%%%%%%%%%%%%%%%%%%%%
%%%%%%%%%%%%%%%%%%%%%%%%%%%%%%%%%%%%%%%%%%%%%%%%%%%%%%%%%%%%%%%%%%%%%%%%%%%%%%%
%%%%%%%%%%%%%%%%%%%%%%%%%%%%%%%%%%%%%%%%%%%%%%%%%%%%%%%%%%%%%%%%%%%%%%%%%%%%%%%
\subsection{Etichette e riferimenti incrociati}
\begin{frame}[fragile]{\insertsubsection}
\begin{itemize}{\small
\item Usate i comandi \cmdbs{label} e \cmdbs{ref} per la numerazione automatica
e i riferimenti incrociati.
\item Il pacchetto \bftt{amsmath} offre il comando \cmdbs{eqref} per numerare
le equazioni.
}\end{itemize}
\begin{minipage}{0.50\linewidth}
\inputminted[fontsize=\scriptsize,frame=single,resetmargins]{latex}%
  {structure-crossref.tex}
\end{minipage}
\begin{minipage}{0.45\linewidth}
\includegraphics[width=\textwidth,clip,trim=1.5in 6in 1.6in 1in]{structure-crossref.pdf}
\end{minipage}
\end{frame}

%%%%%%%%%%%%%%%%%%%%%%%%%%%%%%%%%%%%%%%%%%%%%%%%%%%%%%%%%%%%%%%%%%%%%%%%%%%%%%%
%%%%%%%%%%%%%%%%%%%%%%%%%%%%%%%%%%%%%%%%%%%%%%%%%%%%%%%%%%%%%%%%%%%%%%%%%%%%%%%
%%%%%%%%%%%%%%%%%%%%%%%%%%%%%%%%%%%%%%%%%%%%%%%%%%%%%%%%%%%%%%%%%%%%%%%%%%%%%%%
\subsection{Esercizio}
\begin{frame}[fragile]{Esercizio sulla struttura dei documenti}

\begin{block}{Scrivi questo breve paper in \LaTeX\footnote{Da \url{http://pdos.csail.mit.edu/scigen/}, un generatore casuale di paper.}:}
\begin{center}
\fbox{\href{\fileuri/structure-exercise-solution.pdf}{%
Clicca per aprire il paper}}
\end{center}
Cercate di rendere il paper simile a quello dell'esempio. Usate \cmdbs{ref} e \cmdbs{eqref} per evitare di scrivere esplicitamente nel testo i numeri
di sezione e delle equazioni.
\end{block}
\vskip 2ex
\begin{center}
\fbox{\href{\wlnewdoc{structure-exercise.tex}}{%
Clicca per aprire questo esercizio con \wllogo{}}}
\end{center}

\begin{itemize}
\item Dopo qualche tentativo,
\fbox{\href{\wlnewdoc{structure-exercise-solution.tex}}{%
clicca qui per la mia soluzione}}.
\end{itemize}
\end{frame}

%%%%%%%%%%%%%%%%%%%%%%%%%%%%%%%%%%%%%%%%%%%%%%%%%%%%%%%%%%%%%%%%%%%%%%%%%%%%%%%
%%%%%%%%%%%%%%%%%%%%%%%%%%%%%%%%%%%%%%%%%%%%%%%%%%%%%%%%%%%%%%%%%%%%%%%%%%%%%%%
%%%%%%%%%%%%%%%%%%%%%%%%%%%%%%%%%%%%%%%%%%%%%%%%%%%%%%%%%%%%%%%%%%%%%%%%%%%%%%%
\section{Immagini e tabelle}

%%%%%%%%%%%%%%%%%%%%%%%%%%%%%%%%%%%%%%%%%%%%%%%%%%%%%%%%%%%%%%%%%%%%%%%%%%%%%%%
%%%%%%%%%%%%%%%%%%%%%%%%%%%%%%%%%%%%%%%%%%%%%%%%%%%%%%%%%%%%%%%%%%%%%%%%%%%%%%%
%%%%%%%%%%%%%%%%%%%%%%%%%%%%%%%%%%%%%%%%%%%%%%%%%%%%%%%%%%%%%%%%%%%%%%%%%%%%%%%
\begin{frame}{Indice}
\begin{multicols}{2}
\tableofcontents[currentsection]
\end{multicols}
\end{frame}

%%%%%%%%%%%%%%%%%%%%%%%%%%%%%%%%%%%%%%%%%%%%%%%%%%%%%%%%%%%%%%%%%%%%%%%%%%%%%%%
%%%%%%%%%%%%%%%%%%%%%%%%%%%%%%%%%%%%%%%%%%%%%%%%%%%%%%%%%%%%%%%%%%%%%%%%%%%%%%%
%%%%%%%%%%%%%%%%%%%%%%%%%%%%%%%%%%%%%%%%%%%%%%%%%%%%%%%%%%%%%%%%%%%%%%%%%%%%%%%
\subsection{Grafica}
\begin{frame}[fragile]{\insertsubsection}
\begin{itemize}
\item Per inserire immagini nel testo, serve il pacchetto \bftt{graphicx},
che offre il comando \cmdbs{includegraphics}.
\item I formati supportati per le immagini includono (solitamente)
JPEG, PNG e PDF. Altri pacchetti supportano altri formati.
\end{itemize}
\begin{exampletwouptiny}
\includegraphics[
  width=0.5\textwidth]{pulcino_grande}

\includegraphics[
  width=0.3\textwidth,
  angle=270]{pulcino_grande}
\end{exampletwouptiny}

\tiny{Image from \url{http://www.andy-roberts.net/writing/latex/importing_images}}
\end{frame}

%%%%%%%%%%%%%%%%%%%%%%%%%%%%%%%%%%%%%%%%%%%%%%%%%%%%%%%%%%%%%%%%%%%%%%%%%%%%%%%
%%%%%%%%%%%%%%%%%%%%%%%%%%%%%%%%%%%%%%%%%%%%%%%%%%%%%%%%%%%%%%%%%%%%%%%%%%%%%%%
%%%%%%%%%%%%%%%%%%%%%%%%%%%%%%%%%%%%%%%%%%%%%%%%%%%%%%%%%%%%%%%%%%%%%%%%%%%%%%%
\begin{frame}[fragile]{Intermezzo: Argomenti Opzionali}
\begin{itemize}
\item Si usano le parentesi quadre \keystrokebftt{[} \keystrokebftt{]} per gli
argomenti opzionali, invece che le graffe \keystrokebftt{\{} \keystrokebftt{\}}.
\item \cmdbs{includegraphics} accetta una serie di opzioni che permettono di
trasformare l'immagine quando viene inclusa nel testo. Per esempio, \bftt{width=0.3\cmdbs{textwidth}} fa s\`i che l'immagine sia larga
quanto il 30\% del testo circostante\\(il cui valore \`e contenuto in \cmdbs{textwidth}).
\item Anche \cmdbs{documentclass} accetta opzioni. Per esempio:
\vskip 1.5ex
\mint{latex}|\documentclass[12pt,twocolumn]{article}|
\vskip 1.5ex
usa un font pi\`u grande di quello standard (12pt) e un layout a due colonne.
\item Come scoprire quali argomenti opzionali sono disponibili?
Alla fine della presentazione, mostrer\`o alcuni link\ldots
\end{itemize}
\end{frame}

%%%%%%%%%%%%%%%%%%%%%%%%%%%%%%%%%%%%%%%%%%%%%%%%%%%%%%%%%%%%%%%%%%%%%%%%%%%%%%%
%%%%%%%%%%%%%%%%%%%%%%%%%%%%%%%%%%%%%%%%%%%%%%%%%%%%%%%%%%%%%%%%%%%%%%%%%%%%%%%
%%%%%%%%%%%%%%%%%%%%%%%%%%%%%%%%%%%%%%%%%%%%%%%%%%%%%%%%%%%%%%%%%%%%%%%%%%%%%%%
\subsection[fragile]{Flottanti}
\begin{frame}{\insertsubsection}
\begin{itemize}
\item Permettono a \LaTeX{} di decidere il posizionamento della figura (potr\`a `flottare' -- o galleggiare -- nel testo).
\item Cos\`i facendo \`e anche possibile dare didascalie alle figure, che possono
essere richiamate con \cmdbs{ref}.
\end{itemize}
\begin{minipage}{0.55\linewidth}
\inputminted[fontsize=\scriptsize,frame=single,resetmargins]{latex}%
  {media-graphics.tex}
\end{minipage}
\begin{minipage}{0.35\linewidth}
\includegraphics[width=\textwidth,clip,trim=2in 5in 3in 1in]{media-graphics.pdf}
\end{minipage}
\end{frame}

%%%%%%%%%%%%%%%%%%%%%%%%%%%%%%%%%%%%%%%%%%%%%%%%%%%%%%%%%%%%%%%%%%%%%%%%%%%%%%%
%%%%%%%%%%%%%%%%%%%%%%%%%%%%%%%%%%%%%%%%%%%%%%%%%%%%%%%%%%%%%%%%%%%%%%%%%%%%%%%
%%%%%%%%%%%%%%%%%%%%%%%%%%%%%%%%%%%%%%%%%%%%%%%%%%%%%%%%%%%%%%%%%%%%%%%%%%%%%%%
\subsection{Tabelle}
\begin{frame}[fragile]{\insertsubsection}
\begin{itemize}
\item Si usa l'ambiente \bftt{tabular} dal pacchetto \bftt{tabularx}.
\item L'argomento imposta l'allineamento delle colunne --\\
\textbf{l}eft, \textbf{r}ight, \textbf{r}ight.
\begin{exampletwouptiny}
\begin{tabular}{lrr}
Art.   & Num & \euro \\
Coso   & 1   & 199.99  \\
Gadget & 2   & 399.99  \\
Cavo   & 3   & 19.99   \\
\end{tabular}
\end{exampletwouptiny}
\item L'argomento permette anche di specificare linee verticali;\\per quelle orizzontali si usa \cmdbs{hline}.
\begin{exampletwouptiny}
\begin{tabular}{|l|r|r|} \hline
Art.   & Num & \euro   \\\hline
Coso   & 1   & 199.99  \\
Gadget & 2   & 399.99  \\
Cavo   & 3   & 19.99   \\\hline
\end{tabular}
\end{exampletwouptiny}
\item Usa l'ampersand \keystrokebftt{\&} per separare le colonne e un doppio backslash \keystrokebftt{\bs\bs} per iniziare una nuova riga
(come nell'ambiente \bftt{align*} che abbiamo visto nella Parte 1).
\end{itemize}
\end{frame}

%%%%%%%%%%%%%%%%%%%%%%%%%%%%%%%%%%%%%%%%%%%%%%%%%%%%%%%%%%%%%%%%%%%%%%%%%%%%%%%
%%%%%%%%%%%%%%%%%%%%%%%%%%%%%%%%%%%%%%%%%%%%%%%%%%%%%%%%%%%%%%%%%%%%%%%%%%%%%%%
%%%%%%%%%%%%%%%%%%%%%%%%%%%%%%%%%%%%%%%%%%%%%%%%%%%%%%%%%%%%%%%%%%%%%%%%%%%%%%%
\addtocontents{toc}{\newpage}
\section{Bibliografia}

%%%%%%%%%%%%%%%%%%%%%%%%%%%%%%%%%%%%%%%%%%%%%%%%%%%%%%%%%%%%%%%%%%%%%%%%%%%%%%%
%%%%%%%%%%%%%%%%%%%%%%%%%%%%%%%%%%%%%%%%%%%%%%%%%%%%%%%%%%%%%%%%%%%%%%%%%%%%%%%
%%%%%%%%%%%%%%%%%%%%%%%%%%%%%%%%%%%%%%%%%%%%%%%%%%%%%%%%%%%%%%%%%%%%%%%%%%%%%%%
\begin{frame}{Indice}
\begin{multicols}{2}
\tableofcontents[currentsection]
\end{multicols}
\end{frame}

%%%%%%%%%%%%%%%%%%%%%%%%%%%%%%%%%%%%%%%%%%%%%%%%%%%%%%%%%%%%%%%%%%%%%%%%%%%%%%%
%%%%%%%%%%%%%%%%%%%%%%%%%%%%%%%%%%%%%%%%%%%%%%%%%%%%%%%%%%%%%%%%%%%%%%%%%%%%%%%
%%%%%%%%%%%%%%%%%%%%%%%%%%%%%%%%%%%%%%%%%%%%%%%%%%%%%%%%%%%%%%%%%%%%%%%%%%%%%%%
\subsection{bib\TeX}
\begin{frame}[fragile]{\insertsubsection{} 1}
\begin{itemize}
\item I riferimenti bibliografici andrebbero messi in un file \bftt{.bib}
usando il formato `bibtex':
\inputminted[fontsize=\scriptsize,frame=single]{latex}{bib-example.bib}
\item La maggior parte dei motori di ricerca permettono di esportare
direttamente in formato bibtex
\end{itemize}
\end{frame}

%%%%%%%%%%%%%%%%%%%%%%%%%%%%%%%%%%%%%%%%%%%%%%%%%%%%%%%%%%%%%%%%%%%%%%%%%%%%%%%
%%%%%%%%%%%%%%%%%%%%%%%%%%%%%%%%%%%%%%%%%%%%%%%%%%%%%%%%%%%%%%%%%%%%%%%%%%%%%%%
%%%%%%%%%%%%%%%%%%%%%%%%%%%%%%%%%%%%%%%%%%%%%%%%%%%%%%%%%%%%%%%%%%%%%%%%%%%%%%%
\begin{frame}[fragile]{\insertsubsection{} 2}
\begin{itemize}
\item Ogni elemento in un file \bftt{.bib} ha una \structure{chiave} che si usa per farne riferimento nel testo. Per esempio, \bftt{Scarson1999Stuff} \`e la chiave per l'articolo:
\begin{minted}[fontsize=\small,frame=single]{latex}
@Article{Scarson1999Stuff,
  author = {Von Scarson},
  ...
}
\end{minted}
\item Non \`e obbligatorio, ma una buona idea \`e di usare una chiave basata su
nome, anno e titolo del paper.
\item \LaTeX{} formatta in automatico le citazioni nel testo, e genera una bibliografia: sono disponibili tutti gli stili pi\`u comuni, e se ne possono generare di personalizzati.
\end{itemize}
\end{frame}

%%%%%%%%%%%%%%%%%%%%%%%%%%%%%%%%%%%%%%%%%%%%%%%%%%%%%%%%%%%%%%%%%%%%%%%%%%%%%%%
%%%%%%%%%%%%%%%%%%%%%%%%%%%%%%%%%%%%%%%%%%%%%%%%%%%%%%%%%%%%%%%%%%%%%%%%%%%%%%%
%%%%%%%%%%%%%%%%%%%%%%%%%%%%%%%%%%%%%%%%%%%%%%%%%%%%%%%%%%%%%%%%%%%%%%%%%%%%%%%
\begin{frame}[fragile]{\insertsubsection{} 3}
\begin{itemize}
\item Usate il pacchetto \bftt{natbib}\footnote{Il pacchetto \bftt{biblatex}
\`e pi\`u recente e potente ma \bftt{natbib} \`e ancora il pi\`u diffuso e
usato in molti \emph{template} di riviste.} con \cmdbs{citet} e \cmdbs{citep}.
\item Includete la bibliografia con il comando \cmdbs{bibliography} alla fine, e specificate uno stile con \cmdbs{bibliographystyle}.
\end{itemize}
\begin{minipage}{0.55\linewidth}
\inputminted[fontsize=\scriptsize,frame=single,resetmargins]{latex}%
  {bib-example.tex}
\end{minipage}
\begin{minipage}{0.35\linewidth}
\includegraphics[width=\textwidth,clip,trim=1.8in 5in 1.8in 1in]{bib-example.pdf}
\end{minipage}
\end{frame}

%%%%%%%%%%%%%%%%%%%%%%%%%%%%%%%%%%%%%%%%%%%%%%%%%%%%%%%%%%%%%%%%%%%%%%%%%%%%%%%
%%%%%%%%%%%%%%%%%%%%%%%%%%%%%%%%%%%%%%%%%%%%%%%%%%%%%%%%%%%%%%%%%%%%%%%%%%%%%%%
%%%%%%%%%%%%%%%%%%%%%%%%%%%%%%%%%%%%%%%%%%%%%%%%%%%%%%%%%%%%%%%%%%%%%%%%%%%%%%%
\subsection{Esercizio}
\begin{frame}[fragile]{Esercizio: Mettiamo tutto insieme}

Aggiungete un'immagine e la bibliografia all'articolo dell'esercizio precedente.

\begin{enumerate}
\item Scarica questi file di esempio sul tuo computer.

\begin{center}
\fbox{\href{\fileuri/pulcino_grande.png?dl=1}
{Clicca per scaricare l'immagine di esempio}}

\fbox{\href{\fileuri/bib-exercise.bib?dl=1}
{Clicca per scaricare il file .bib di esempio}}
\end{center}

\item Caricali su Overleaf (usa il men\`u \structure{Project}).

\end{enumerate}
\end{frame}

%%%%%%%%%%%%%%%%%%%%%%%%%%%%%%%%%%%%%%%%%%%%%%%%%%%%%%%%%%%%%%%%%%%%%%%%%%%%%%%
%%%%%%%%%%%%%%%%%%%%%%%%%%%%%%%%%%%%%%%%%%%%%%%%%%%%%%%%%%%%%%%%%%%%%%%%%%%%%%%
%%%%%%%%%%%%%%%%%%%%%%%%%%%%%%%%%%%%%%%%%%%%%%%%%%%%%%%%%%%%%%%%%%%%%%%%%%%%%%%
\section{E adesso?}

%%%%%%%%%%%%%%%%%%%%%%%%%%%%%%%%%%%%%%%%%%%%%%%%%%%%%%%%%%%%%%%%%%%%%%%%%%%%%%%
%%%%%%%%%%%%%%%%%%%%%%%%%%%%%%%%%%%%%%%%%%%%%%%%%%%%%%%%%%%%%%%%%%%%%%%%%%%%%%%
%%%%%%%%%%%%%%%%%%%%%%%%%%%%%%%%%%%%%%%%%%%%%%%%%%%%%%%%%%%%%%%%%%%%%%%%%%%%%%%
\begin{frame}{Indice}
\begin{multicols}{2}
\tableofcontents[currentsection]
\end{multicols}
\end{frame}

%%%%%%%%%%%%%%%%%%%%%%%%%%%%%%%%%%%%%%%%%%%%%%%%%%%%%%%%%%%%%%%%%%%%%%%%%%%%%%%
%%%%%%%%%%%%%%%%%%%%%%%%%%%%%%%%%%%%%%%%%%%%%%%%%%%%%%%%%%%%%%%%%%%%%%%%%%%%%%%
%%%%%%%%%%%%%%%%%%%%%%%%%%%%%%%%%%%%%%%%%%%%%%%%%%%%%%%%%%%%%%%%%%%%%%%%%%%%%%%
\subsection{Indici, classi, comandi\ldots}
\begin{frame}[fragile]{\insertsubsection}
\begin{itemize}
\item Aggiungi un indice con il combando \cmdbs{tableofcontents}
a partire dai comandi di sezionamento come \cmdbs{section}.

\item Cambia la classe del documento con \cmdbs{documentclass} a
\mint{latex}!\documentclass{scrartcl}!
o
\mint{latex}!\documentclass[12pt]{IEEEtran}!

\item Definisci comandi personalizzati per un'equazione complessa:
\begin{exampletwouptiny}
\newcommand{\rperf}{%
  \rho_{\text{perf}}}
$$
\rperf = {\bf c}'{\bf X} + \varepsilon
$$
\end{exampletwouptiny}
\end{itemize}
\end{frame}

%%%%%%%%%%%%%%%%%%%%%%%%%%%%%%%%%%%%%%%%%%%%%%%%%%%%%%%%%%%%%%%%%%%%%%%%%%%%%%%
%%%%%%%%%%%%%%%%%%%%%%%%%%%%%%%%%%%%%%%%%%%%%%%%%%%%%%%%%%%%%%%%%%%%%%%%%%%%%%%
%%%%%%%%%%%%%%%%%%%%%%%%%%%%%%%%%%%%%%%%%%%%%%%%%%%%%%%%%%%%%%%%%%%%%%%%%%%%%%%
\subsection{Qualche esempio di pacchetto}
\begin{frame}{\insertsubsection}
\begin{itemize}
\item \bftt{beamer}: creazione di presentazioni (come questa!)
\item \bftt{todonotes}: gestione commenti e TODO
\item \bftt{tikz}: gestione della grafica
\item \bftt{pgfplots}: per creare grafi in \LaTeX
\item \bftt{listings}: per la composizione di codici sorgente
\item \bftt{spreadtab}: creazione fogli di calcolo in \LaTeX
\item \bftt{gchords}, \bftt{guitar}: spartiti e accordi per chitarra
\item \bftt{cwpuzzle}: parole crociate
\end{itemize}
Vai su \url{https://www.overleaf.com/latex/examples} e \url{http://texample.net}
per trovare esempi che fanno uso di questi pacchetti\ldots
\end{frame}

%%%%%%%%%%%%%%%%%%%%%%%%%%%%%%%%%%%%%%%%%%%%%%%%%%%%%%%%%%%%%%%%%%%%%%%%%%%%%%%
%%%%%%%%%%%%%%%%%%%%%%%%%%%%%%%%%%%%%%%%%%%%%%%%%%%%%%%%%%%%%%%%%%%%%%%%%%%%%%%
%%%%%%%%%%%%%%%%%%%%%%%%%%%%%%%%%%%%%%%%%%%%%%%%%%%%%%%%%%%%%%%%%%%%%%%%%%%%%%%
\subsection{Installare \LaTeX{}}
\begin{frame}{\insertsubsection}
\begin{itemize}
\item Per eseguire \LaTeX{} sul vostro computer, vi servir\`a una \structure{distribuzione} \LaTeX{}. Una distribuzione include il comando \bftt{latex} 
e qualche migliaio di pacchetti.
\begin{itemize}
\item Su Windows: \href{http://miktex.org/}{Mik\TeX} o \href{http://tug.org/texlive/}{\TeX Live}
\item Su Linux: \href{http://tug.org/texlive/}{\TeX Live}
\item SU Mac: \href{http://tug.org/mactex/}{Mac\TeX}
\end{itemize}
\item Vi servir\`a anche un editor testuale con supporto \LaTeX{}. \url{http://en.wikipedia.org/wiki/Comparison_of_TeX_editors} compara tutte le possibili opzioni.
\item Vorrete saperne di pi\`u sul funzionamento di \bftt{latex} e degli strumenti
correlati --- alcune risorse sono sulla prossima slide.
\end{itemize}
\end{frame}

%%%%%%%%%%%%%%%%%%%%%%%%%%%%%%%%%%%%%%%%%%%%%%%%%%%%%%%%%%%%%%%%%%%%%%%%%%%%%%%
%%%%%%%%%%%%%%%%%%%%%%%%%%%%%%%%%%%%%%%%%%%%%%%%%%%%%%%%%%%%%%%%%%%%%%%%%%%%%%%
%%%%%%%%%%%%%%%%%%%%%%%%%%%%%%%%%%%%%%%%%%%%%%%%%%%%%%%%%%%%%%%%%%%%%%%%%%%%%%%
\subsection{Risorse addizionali}
\begin{frame}{\insertsubsection}
\begin{itemize}
\item \href{http://en.wikibooks.org/wiki/LaTeX}{The \LaTeX{} Wikibook} ---
materiale di riferimento ed eccellenti tutorial.
\item \href{http://tex.stackexchange.com/}{\TeX{} Stack Exchange} --- fai domani e ottieni ottime risposte in pochissimo tempo
\item \href{http://www.latex-community.org/}{\LaTeX{} Community} --- il principale forum degli utilizzatori
\item \href{http://ctan.org/}{Comprehensive \TeX{} Archive Network (CTAN)} ---
oltre quattromila pacchetti e rispettiva documentazione
\item Una ricerca diretta su Google\ldots
\end{itemize}
\end{frame}

%%%%%%%%%%%%%%%%%%%%%%%%%%%%%%%%%%%%%%%%%%%%%%%%%%%%%%%%%%%%%%%%%%%%%%%%%%%%%%%
%%%%%%%%%%%%%%%%%%%%%%%%%%%%%%%%%%%%%%%%%%%%%%%%%%%%%%%%%%%%%%%%%%%%%%%%%%%%%%%
%%%%%%%%%%%%%%%%%%%%%%%%%%%%%%%%%%%%%%%%%%%%%%%%%%%%%%%%%%%%%%%%%%%%%%%%%%%%%%%
\begin{frame}
\begin{center}
Grazie, e buon lavoro con \LaTeX{}!
\end{center}
\end{frame}

\end{document}

Contenuto addizionale:

Emphasized text is typed like this: \emph{this is emphasized}.
Bold       text is typed like this: \textbf{this is bold}.

Riferimenti:

--> maybe introduce the prettyref package here.

Tabelle:

Bonus points: check out the fp package and the spreadtab package.

Document Classes:

a .cls file (es. article) some journal templates come with one

-- For Typesetting Geeks

- dashes: -, --, ---
- ellipsis.
- controlling spaces: ~, \ , \,, \@
- spacing after periods (et al., etc.)
- Nested quotation marks: ``\,`

\begin{center}
\fbox{\href{http://ctan.org/}{The Comprehensive \TeX Archive Network (CTAN)}}
\end{center}