\documentclass{beamer}

\input{preamble.tex}

\subtitle{Parte 1: Le basi}

\begin{document}

%%%%%%%%%%%%%%%%%%%%%%%%%%%%%%%%%%%%%%%%%%%%%%%%%%%%%%%%%%%%%%%%%%%%%%%%%%%%%%%
%%%%%%%%%%%%%%%%%%%%%%%%%%%%%%%%%%%%%%%%%%%%%%%%%%%%%%%%%%%%%%%%%%%%%%%%%%%%%%%
%%%%%%%%%%%%%%%%%%%%%%%%%%%%%%%%%%%%%%%%%%%%%%%%%%%%%%%%%%%%%%%%%%%%%%%%%%%%%%%
\begin{frame}
\titlepage
\end{frame}

%%%%%%%%%%%%%%%%%%%%%%%%%%%%%%%%%%%%%%%%%%%%%%%%%%%%%%%%%%%%%%%%%%%%%%%%%%%%%%%
%%%%%%%%%%%%%%%%%%%%%%%%%%%%%%%%%%%%%%%%%%%%%%%%%%%%%%%%%%%%%%%%%%%%%%%%%%%%%%%
%%%%%%%%%%%%%%%%%%%%%%%%%%%%%%%%%%%%%%%%%%%%%%%%%%%%%%%%%%%%%%%%%%%%%%%%%%%%%%%
\section{Introduzione}
\begin{frame}{Perch\'e \LaTeX{}?}
\begin{itemize}
\item Permette di realizzare documenti professionali e esteticamente appaganti
\begin{itemize}
\item Specialmente se contengono \structure{matematica}
\end{itemize}
%
\item \`E stato creato da scienziati, per scienziati
\begin{itemize}
\item Una comunit\`a enorme e molto attiva
\item \ldots ma il suo uso \`e estendibile ad ogni campo
\end{itemize}
%
\item \`E potentissimo --- ed estendibile a piacimento
\begin{itemize}
\item \structure{Pacchetti} per articoli scientifici, libri, presentazioni,
fogli di calcolo, \ldots
\end{itemize}
\end{itemize}
\end{frame}

%%%%%%%%%%%%%%%%%%%%%%%%%%%%%%%%%%%%%%%%%%%%%%%%%%%%%%%%%%%%%%%%%%%%%%%%%%%%%%%
%%%%%%%%%%%%%%%%%%%%%%%%%%%%%%%%%%%%%%%%%%%%%%%%%%%%%%%%%%%%%%%%%%%%%%%%%%%%%%%
%%%%%%%%%%%%%%%%%%%%%%%%%%%%%%%%%%%%%%%%%%%%%%%%%%%%%%%%%%%%%%%%%%%%%%%%%%%%%%%
\begin{frame}{Esempi}
\begin{equation}
\sum_{n=1}^k \frac{1}{n} ≻ \int_1^{k+1} \frac{1}{x} dx = \ln(k+1)
\end{equation}
\begin{equation}
\sum_{n = 0}^\infty \frac{(-1)^{n}}{2n+1} = 1 - \frac{1}{3} + \frac{1}{5} - \frac{1}{7} + \cdots = \frac{\pi}{4}
\end{equation}
\begin{equation}
\int_a^b \! f(x) dx = F(b) - F(a)
\end{equation}
\begin{equation}
f'(a)=\lim_{h\to 0}\frac{f(a+h)-f(a)}{h}
\end{equation}
\begin{equation}
\textstyle r=|z|=\sqrt{x^2+y^2}
\end{equation}
\begin{equation}
\begin{bmatrix} 1 & 2\\ 3 & 4\\ \end{bmatrix} \begin{bmatrix} 0 & 1\\ 0 & 0\\ \end{bmatrix}= \begin{bmatrix} 0 & 1\\ 0 & 3\\ \end{bmatrix}
\end{equation}
\end{frame}

\begin{frame}{Esempi}
\shapepar{\heartshape}
If I speak in the tongues of men or of angels, but do not have love, I am only a resounding gong or a clanging cymbal. If I have the gift of prophecy and can fathom all mysteries and all knowledge, and if I have a faith that can move mountains, but do not have love, I am nothing.  If I give all I possess to the poor and give over my body to hardship that I may boast, but do not have love, I gain nothing. Love is patient, love is kind. It does not envy, it does not boast, it is not proud.  It does not dishonor others, it is not self-seeking, it is not easily angered, it keeps no record of wrongs.  Love does not delight in evil but rejoices with the truth. It always protects, always trusts, always hopes, always perseveres.
\end{frame}

\begin{frame}{Esempi}
\bigskip\bigskip\bigskip
\foreignlanguage{polutonikogreek}{
\begin{verse}Th| p'anta dido'ush| ka`i >apolambano'ush| f'usei <o pepaideum'enos ka`i a>id'hmwn l'egei; <<d`os, <`o j'eleis, >ap'olabe, <`o j'eleic>>.
L'egei d`e to~uto o>u katajrasun'omenos, >all`a peijarq~wn m'onon ka`i e>uno~wn a>ut~h|.
\end{verse}}
\begin{flushright}
- Marco Aurelio, \emph{Ricordi}
\end{flushright}
\end{frame}

%%%%%%%%%%%%%%%%%%%%%%%%%%%%%%%%%%%%%%%%%%%%%%%%%%%%%%%%%%%%%%%%%%%%%%%%%%%%%%%
%%%%%%%%%%%%%%%%%%%%%%%%%%%%%%%%%%%%%%%%%%%%%%%%%%%%%%%%%%%%%%%%%%%%%%%%%%%%%%%
%%%%%%%%%%%%%%%%%%%%%%%%%%%%%%%%%%%%%%%%%%%%%%%%%%%%%%%%%%%%%%%%%%%%%%%%%%%%%%%
\begin{frame}{Come si pronuncia?}

\begin{itemize}
\item \TeX{} \`e stato creato a Stanford da Donald E. Knuth
\item Il nome deriva dalla radice greca di parole come
\textbf{$\tau\epsilon\chi\nu\acute{\eta}$}
che significa \emph{arte} o \emph{tecnica}.
\begin{itemize}
\item la pronuncia dovrebbe essere \emph{tech} (come il tedesco \emph{Bach})
\item in italiano solitamente si usa \emph{tek}
\end{itemize}
\item \LaTeX{} \`e un set di macro per \TeX creato da Leslie Lamport

\begin{itemize}
\item oramai nessuno usa pi\`u \TeX base
\item la pronuncia usuale italiana \`e \emph{latek}.
\end{itemize}
\end{itemize}
\end{frame}

%%%%%%%%%%%%%%%%%%%%%%%%%%%%%%%%%%%%%%%%%%%%%%%%%%%%%%%%%%%%%%%%%%%%%%%%%%%%%%%
%%%%%%%%%%%%%%%%%%%%%%%%%%%%%%%%%%%%%%%%%%%%%%%%%%%%%%%%%%%%%%%%%%%%%%%%%%%%%%%
%%%%%%%%%%%%%%%%%%%%%%%%%%%%%%%%%%%%%%%%%%%%%%%%%%%%%%%%%%%%%%%%%%%%%%%%%%%%%%%

\begin{frame}[fragile]{Come funziona?}
\begin{itemize}
\item Un documento \LaTeX \`e composto di \texttt{testo semplice} inframezzato a
\cmd{comandi} che ne descrivono la struttura e il significato.
\item L'applicazione \texttt{latex} \alert{compila} il testo e i comandi per produrre un documento perfettamente formattato.
\end{itemize}
\vskip 2ex
\begin{center}
\begin{minted}[frame=single]{latex}
La rana in Spagna \emph{gracida} in campagna.
\end{minted}
\vskip 2ex
\tikz\node[single arrow,fill=gray,font=\ttfamily\bfseries,%
  rotate=270,xshift=-1em]{latex};
\vskip 2ex
\fbox{La rana in Spagna \emph{gracida} in campagna.}
\end{center}
\end{frame}

%%%%%%%%%%%%%%%%%%%%%%%%%%%%%%%%%%%%%%%%%%%%%%%%%%%%%%%%%%%%%%%%%%%%%%%%%%%%%%%
%%%%%%%%%%%%%%%%%%%%%%%%%%%%%%%%%%%%%%%%%%%%%%%%%%%%%%%%%%%%%%%%%%%%%%%%%%%%%%%
%%%%%%%%%%%%%%%%%%%%%%%%%%%%%%%%%%%%%%%%%%%%%%%%%%%%%%%%%%%%%%%%%%%%%%%%%%%%%%%
\subsection{Compila?}

\begin{frame}[fragile]{\insertsubsection{} Un classico programma C}

Facciamo un paragone con il signor Pippo che vuole scrivere un programma
nel suo linguaggio preferito, C

\bigskip{}
\begin{enumerate}
\item Pippo scrive un documento di testo che chiamer\`a \cmd{pippo.c}.
\item \label{enu:compila}Pippo d\`a sul terminale il comando
\begin{minted}[frame=single]{bash}
gcc pippo.c
\end{minted}
\item Se la compilazione d\`a errori, Pippo rivede il programma, lo corregge
e ritorna al passo \ref{enu:compila}.
\item Se la compilazione ha successo, viene prodotto un file eseguibile,
\cmd{a.out}.
\item Pippo prova a vedere se il programma fa ci\`o che desidera.
\end{enumerate}
\end{frame}

%%%%%%%%%%%%%%%%%%%%%%%%%%%%%%%%%%%%%%%%%%%%%%%%%%%%%%%%%%%%%%%%%%%%%%%%%%%%%%%
%%%%%%%%%%%%%%%%%%%%%%%%%%%%%%%%%%%%%%%%%%%%%%%%%%%%%%%%%%%%%%%%%%%%%%%%%%%%%%%
%%%%%%%%%%%%%%%%%%%%%%%%%%%%%%%%%%%%%%%%%%%%%%%%%%%%%%%%%%%%%%%%%%%%%%%%%%%%%%%
\begin{frame}[fragile]{\insertsubsection{} Il motore \LaTeX}

Ora Pippo vuole scrivere la documentazione per il suo programma.
\bigskip
\begin{enumerate}
\item Pippo scrive un documento di testo che chiamer\`a \cmd{pippo.tex}.
\item \label{enu:compilalatex}Pippo d\`a sul terminale il comando

\begin{minted}[frame=single]{bash}
latex pippo.tex
\end{minted}

\item Se la compilazione d\`a errori, Pippo rivede il documento, lo corregge
e ritorna al passo \ref{enu:compilalatex}.
\item Se la compilazione ha successo, viene prodotto un file \cmd{pippo.dvi}.
\item Pippo chiama il visualizzatore con

\begin{minted}[frame=single]{bash}
xdvi pippo
\end{minted}

e controlla che non ci siano errori concettuali.
\end{enumerate}
\end{frame}

%%%%%%%%%%%%%%%%%%%%%%%%%%%%%%%%%%%%%%%%%%%%%%%%%%%%%%%%%%%%%%%%%%%%%%%%%%%%%%%
%%%%%%%%%%%%%%%%%%%%%%%%%%%%%%%%%%%%%%%%%%%%%%%%%%%%%%%%%%%%%%%%%%%%%%%%%%%%%%%
%%%%%%%%%%%%%%%%%%%%%%%%%%%%%%%%%%%%%%%%%%%%%%%%%%%%%%%%%%%%%%%%%%%%%%%%%%%%%%%
\begin{frame}[fragile]{\insertsubsection{} pdf\LaTeX}

Ora Pippo vuole scrivere la documentazione per il suo programma in
un formato pi\`u comune.
\bigskip
\begin{enumerate}
\item Pippo ha gi\`a il documento di testo chiamato \cmd{pippo.tex}.
\item Pippo d\`a sul terminale il comando
\begin{minted}[frame=single]{bash}
pdflatex pippo.tex
\end{minted}
\item Viene prodotto un file \cmd{pippo.pdf}.
\end{enumerate}
\bigskip
Dato che il documento \cmd{pippo.tex} \`e lo stesso di prima,
il documento finale \`e solo la resa in PDF di quello precedente.
\end{frame}

%%%%%%%%%%%%%%%%%%%%%%%%%%%%%%%%%%%%%%%%%%%%%%%%%%%%%%%%%%%%%%%%%%%%%%%%%%%%%%%
%%%%%%%%%%%%%%%%%%%%%%%%%%%%%%%%%%%%%%%%%%%%%%%%%%%%%%%%%%%%%%%%%%%%%%%%%%%%%%%
%%%%%%%%%%%%%%%%%%%%%%%%%%%%%%%%%%%%%%%%%%%%%%%%%%%%%%%%%%%%%%%%%%%%%%%%%%%%%%%
\begin{frame}[fragile]{Alcuni esempi di comandi\ldots}
\begin{exampletwoup}
\begin{itemize}
\item T\'e
\item Latte
\item Biscotti
\end{itemize}
\end{exampletwoup}
\vskip 2ex
\begin{exampletwoup}
\begin{figure}
\includegraphics{pulcino}
\end{figure}
\end{exampletwoup}
\vskip 2ex
\begin{exampletwoup}
\begin{equation}
\alpha + \beta + 1
\end{equation}
\end{exampletwoup}

\tiny{Immagine tratta da \url{http://www.andy-roberts.net/writing/latex/importing_images}}
\end{frame}

%%%%%%%%%%%%%%%%%%%%%%%%%%%%%%%%%%%%%%%%%%%%%%%%%%%%%%%%%%%%%%%%%%%%%%%%%%%%%%%
%%%%%%%%%%%%%%%%%%%%%%%%%%%%%%%%%%%%%%%%%%%%%%%%%%%%%%%%%%%%%%%%%%%%%%%%%%%%%%%
%%%%%%%%%%%%%%%%%%%%%%%%%%%%%%%%%%%%%%%%%%%%%%%%%%%%%%%%%%%%%%%%%%%%%%%%%%%%%%%

\begin{frame}{Distribuzioni di \LaTeX{}}

\begin{block}{Cosa scaricare}

\begin{itemize}
\item GNU/Linux: \TeX{}Live come sorgente o come pacchetto;
\item Mac OS X: Mac\TeX{};
\item Microsoft Windows: Mik\TeX{} (ce ne sono anche altri)
\end{itemize}
\end{block}

\begin{itemize}
\item Ottenuta la distribuzione, il codice si scrive con un editor
	\begin{itemize}
	\item Un \structure{qualsiasi} blocco note serve allo scopo
	\end{itemize}
\item Esistono anche ambienti integrati
	\begin{itemize}
	\item Editor `classici' come \TeX{}Works o \TeX{}Maker
	\item Strumenti `simil-word' come Lyx
	\item \ldots o web application come \wllogo
	\end{itemize}
\item I dettagli sono lasciati al lettore come \alert{facile} esercizio.
\end{itemize}

\end{frame}

%%%%%%%%%%%%%%%%%%%%%%%%%%%%%%%%%%%%%%%%%%%%%%%%%%%%%%%%%%%%%%%%%%%%%%%%%%%%%%%
%%%%%%%%%%%%%%%%%%%%%%%%%%%%%%%%%%%%%%%%%%%%%%%%%%%%%%%%%%%%%%%%%%%%%%%%%%%%%%%
%%%%%%%%%%%%%%%%%%%%%%%%%%%%%%%%%%%%%%%%%%%%%%%%%%%%%%%%%%%%%%%%%%%%%%%%%%%%%%%

\begin{frame}[fragile]{In sintesi: un vero e proprio cambio di paradigma}

\begin{block}{Pregi}
\begin{itemize}
	\item Nelle distribuzioni sono compresi stili per la composizione di articoli o \structure{documenti di livello professionale}.
	\item La composizione di \alert{formule matematiche} \`e di altissimo livello.
	\item L'utente deve concentrarsi solo sul \structure{contenuto del documento e non sulla sua forma finale}.
	\item Si possono generare note a pi\`e di pagina, riferimenti incrociati,
	bibliografie e indici in modo \structure{automatico}.
	\item Migliaia di persone usano \LaTeX: è facile trovare supporto;
	\item \`E gratis.
\end{itemize}
\end{block}

\begin{alertblock}
{Usate i comandi per descrivere `cio che \`e' e non `ci\`o che appare: concentratevi sul contenuto e Lasciate fare a \LaTeX{} il suo lavoro!}
\end{alertblock}
\end{frame}

%%%%%%%%%%%%%%%%%%%%%%%%%%%%%%%%%%%%%%%%%%%%%%%%%%%%%%%%%%%%%%%%%%%%%%%%%%%%%%%
%%%%%%%%%%%%%%%%%%%%%%%%%%%%%%%%%%%%%%%%%%%%%%%%%%%%%%%%%%%%%%%%%%%%%%%%%%%%%%%
%%%%%%%%%%%%%%%%%%%%%%%%%%%%%%%%%%%%%%%%%%%%%%%%%%%%%%%%%%%%%%%%%%%%%%%%%%%%%%%
\section{Le basi}

%%%%%%%%%%%%%%%%%%%%%%%%%%%%%%%%%%%%%%%%%%%%%%%%%%%%%%%%%%%%%%%%%%%%%%%%%%%%%%%
%%%%%%%%%%%%%%%%%%%%%%%%%%%%%%%%%%%%%%%%%%%%%%%%%%%%%%%%%%%%%%%%%%%%%%%%%%%%%%%
%%%%%%%%%%%%%%%%%%%%%%%%%%%%%%%%%%%%%%%%%%%%%%%%%%%%%%%%%%%%%%%%%%%%%%%%%%%%%%%
\subsection{Iniziamo\ldots}
\begin{frame}[fragile]{\insertsubsection}
\begin{itemize}
\item Il documento \LaTeX{} minimale:
\inputminted[frame=single]{latex}{basics.tex}
\item Tutti i comandi iniziano con un \emph{backslash} \keystrokebftt{\bs}.
\item Ogni documenti inizia con un comando \cmdbs{documentclass}.
\item L'\emph{argomento} tra parentesi graffe \keystrokebftt{\{} \keystrokebftt{\}} indica a \LaTeX{} che tipo di documento stiamo creano: un \bftt{article}.
\item Il simbolo di percento \keystrokebftt{\%} d\`a inizio ad un \emph{commento}
--- \LaTeX{} ignorer\`a il resto della riga.
\end{itemize}
\end{frame}

%%%%%%%%%%%%%%%%%%%%%%%%%%%%%%%%%%%%%%%%%%%%%%%%%%%%%%%%%%%%%%%%%%%%%%%%%%%%%%%
%%%%%%%%%%%%%%%%%%%%%%%%%%%%%%%%%%%%%%%%%%%%%%%%%%%%%%%%%%%%%%%%%%%%%%%%%%%%%%%
%%%%%%%%%%%%%%%%%%%%%%%%%%%%%%%%%%%%%%%%%%%%%%%%%%%%%%%%%%%%%%%%%%%%%%%%%%%%%%%

\begin{frame}[fragile]{\insertsubsection{} con \wllogo}
\begin{itemize}
\item Overleaf \`e un webapp per scrivere documenti in \LaTeX.
\item `Compila' un sorgente \LaTeX{} e mostra i risultati in automatico e in
tempo reale.
\vskip 2em
\begin{center}
\fbox{\href{\wlnewdoc{basics.tex}}{%
Clicca qui per aprire il documento di prima con \wllogo{}}}
\\[1ex]\scriptsize{}
Per migliore compatibilit\`a, usate \href{http://www.google.com/chrome}{Chrome} o un  \href{http://www.mozilla.org/en-US/firefox/new/}{FireFox} recente.
\end{center}
\vskip 2ex
\item Nel resto del corso, provate ad eseguire gli esempi, \structure{copiandoli
direttamente su Overleaf}.
\item \textbf{No, davvero, \`e il miglior modo di imparare!}
\end{itemize}
\end{frame}

%%%%%%%%%%%%%%%%%%%%%%%%%%%%%%%%%%%%%%%%%%%%%%%%%%%%%%%%%%%%%%%%%%%%%%%%%%%%%%%
%%%%%%%%%%%%%%%%%%%%%%%%%%%%%%%%%%%%%%%%%%%%%%%%%%%%%%%%%%%%%%%%%%%%%%%%%%%%%%%
%%%%%%%%%%%%%%%%%%%%%%%%%%%%%%%%%%%%%%%%%%%%%%%%%%%%%%%%%%%%%%%%%%%%%%%%%%%%%%%
\subsection{Inserire il testo}
\begin{frame}[fragile]{\insertsubsection{}}
\small
\begin{itemize}
\item Tutto il testo di un qualunque documento va inserito tra \cmdbegin{document}
e \cmdend{document}.
\item Nella maggior parte dei casi, potete inserire testo normalmente.
\begin{exampletwouptiny}
Le parole sono separate da uno
o pi\`u spazi.

I paragrafi sono separati da una
o pi\`u righe vuote.
\end{exampletwouptiny}
\item Lo spazio nel file sorgente viene \emph{aggregato} nell'output.
\begin{exampletwouptiny}
La   rana       in Spagna
gracida in		 montagna.
\end{exampletwouptiny}
\item \keystrokebftt{\bs\bs} forza l'andata a capo, ma non crea un nuovo paragrafo.
\end{itemize}
\end{frame}

%%%%%%%%%%%%%%%%%%%%%%%%%%%%%%%%%%%%%%%%%%%%%%%%%%%%%%%%%%%%%%%%%%%%%%%%%%%%%%%
%%%%%%%%%%%%%%%%%%%%%%%%%%%%%%%%%%%%%%%%%%%%%%%%%%%%%%%%%%%%%%%%%%%%%%%%%%%%%%%
%%%%%%%%%%%%%%%%%%%%%%%%%%%%%%%%%%%%%%%%%%%%%%%%%%%%%%%%%%%%%%%%%%%%%%%%%%%%%%%
\begin{frame}[fragile]{\insertsubsection{}: caratteri speciali}
\small
\begin{itemize}
\item Le virgolette richiedono attenzione: va usato un apostrofo rovesciato \keystroke{\`{}} a sinistra e un apostrofo semplice \keystroke{\'{}} a destra
\begin{exampletwouptiny}
Virgolette semplici: `testo'.

Virgolette doppie: ``testo''.
\end{exampletwouptiny}

\item Alcuni simboli comuni hanno significato speciale in \LaTeX:\\[1ex]
\begin{tabular}{cl|l}
\keystrokebftt{\%} & percento    & commenti \\
\keystrokebftt{\#} & cancelletto & comandi custom \\
\keystrokebftt{\&} & ampersand   & tabelle \\
\keystrokebftt{\$} & dollaro     & matematica \\
\end{tabular}
\item Se provate ad inserirli direttamente, otterrete un messaggio di errore.
Se volete mostrarli nel documento dovete fare \structure{\emph{escape}},
precedendoli con un backslash \keystrokebftt{\bs}\\
\begin{exampletwouptiny}
\$\%\&\#!
\end{exampletwouptiny}
\end{itemize}

\end{frame}

%%%%%%%%%%%%%%%%%%%%%%%%%%%%%%%%%%%%%%%%%%%%%%%%%%%%%%%%%%%%%%%%%%%%%%%%%%%%%%%
%%%%%%%%%%%%%%%%%%%%%%%%%%%%%%%%%%%%%%%%%%%%%%%%%%%%%%%%%%%%%%%%%%%%%%%%%%%%%%%
%%%%%%%%%%%%%%%%%%%%%%%%%%%%%%%%%%%%%%%%%%%%%%%%%%%%%%%%%%%%%%%%%%%%%%%%%%%%%%%

\begin{frame}[fragile]{\insertsubsection{}: gli accenti}
\small
\begin{itemize}
\item Il caratteri base di \LaTeX{} sono per la lingua inglese (ASCII), ma per molte altre lingue, gli accenti sono importanti.
\begin{itemize}
\item \cmd{\bs`} si usa per l'accento grave
\item \cmd{\bs'} si usa per l'accento acuto.
\item \ldots in alternativa si possono inserire direttamente da tastiera
\begin{minted}{latex}
\usepackage[latin1]{inputenc}
\end{minted}
\end{itemize}
\item Esempio:
\begin{exampletwouptiny}
R\`{e}n\'{e} Descartes \'{e} noto
alla latina come `Cartesio', 
fu scienziato e ``filosofo''.
Mor\`{i} a Stoccolma di polmonite.
\end{exampletwouptiny}

\item Altri glifi non presenti in italiano, e relativi comandi:
\begin{exampletwouptiny}
Dieresi: Fl\"ugel\\
Circonflesso: H\^opital\\
Tilde: Vamos a ga\~nar\\
\end{exampletwouptiny}
\end{itemize}
\end{frame}

%%%%%%%%%%%%%%%%%%%%%%%%%%%%%%%%%%%%%%%%%%%%%%%%%%%%%%%%%%%%%%%%%%%%%%%%%%%%%%%
%%%%%%%%%%%%%%%%%%%%%%%%%%%%%%%%%%%%%%%%%%%%%%%%%%%%%%%%%%%%%%%%%%%%%%%%%%%%%%%
%%%%%%%%%%%%%%%%%%%%%%%%%%%%%%%%%%%%%%%%%%%%%%%%%%%%%%%%%%%%%%%%%%%%%%%%%%%%%%%
\begin{frame}[fragile]{Gestire gli errori}
\begin{itemize}
\item \LaTeX{} pu\`o confondersi nel compilare un documento.
Se succede, si interrompe con un \structure{messaggio di errore}.
\item Dovete correggere gli errori, se volete avere qualche speranza di
produrre un documento.
\item Per esempio, se provate a scrivere \cmdbs{epmh} invece di \cmdbs{emph},
\LaTeX{} si lamenter\`a con un errore \texttt{undefined control sequence} dato che ``epmh'' non esiste come comando.
\end{itemize}
\begin{block}{Qualche consiglio sugli errori}
\begin{enumerate}
\item Niente panico! Succede a tutti.
\item Correggeli immediatamente --- se quello che avete appena scritto ha causato
un errore, perlomeno sapete da dove partire per il debugging.
\item Se ci sono errori multipli, correggeteli uno alla volta iniziando dal primo 
--- potrebbero essere errori a cascata.
\end{enumerate}
\end{block}
\end{frame}

%%%%%%%%%%%%%%%%%%%%%%%%%%%%%%%%%%%%%%%%%%%%%%%%%%%%%%%%%%%%%%%%%%%%%%%%%%%%%%%
%%%%%%%%%%%%%%%%%%%%%%%%%%%%%%%%%%%%%%%%%%%%%%%%%%%%%%%%%%%%%%%%%%%%%%%%%%%%%%%
%%%%%%%%%%%%%%%%%%%%%%%%%%%%%%%%%%%%%%%%%%%%%%%%%%%%%%%%%%%%%%%%%%%%%%%%%%%%%%%
\begin{frame}[fragile]{Esercizio 1}

\begin{block}{Scrivi questo in \LaTeX{}:%
\footnote{\url{http://en.wikipedia.org/wiki/Economy_of_the_United_States}}}
Nel Marzo 2006, il Congresso aument\`o la soglia di
\$790 miliardi per un totale di \$8970 miliardi,
che rappresentava circa il 68\% del PIL. Il 4 Ottobre
2008, l'``Emergency Economic Stabilization
Act'' aument\`o ulteriormente il tetto del debito
a \$11300 miliardi.
\end{block}
\vskip 2ex
\begin{center}
\fbox{\href{\wlnewdoc{basics-exercise-1.tex}}{%
Clicca per aprire l'esercizio in \wllogo{}}}
\end{center}

\begin{itemize}
\item Suggerimento: attenti ai caratteri speciali!
\item Non dimenticate virgolette e accenti \ldots
\item Dopo qualche tentativo,
\fbox{\href{\wlnewdoc{basics-exercise-1-solution.tex}}{%
cliccate qui per la soluzione}}.
\end{itemize}
\end{frame}

%%%%%%%%%%%%%%%%%%%%%%%%%%%%%%%%%%%%%%%%%%%%%%%%%%%%%%%%%%%%%%%%%%%%%%%%%%%%%%%
%%%%%%%%%%%%%%%%%%%%%%%%%%%%%%%%%%%%%%%%%%%%%%%%%%%%%%%%%%%%%%%%%%%%%%%%%%%%%%%
%%%%%%%%%%%%%%%%%%%%%%%%%%%%%%%%%%%%%%%%%%%%%%%%%%%%%%%%%%%%%%%%%%%%%%%%%%%%%%%
\subsection{Ambienti matematici}
\begin{frame}[fragile]{\insertsubsection{}: il dollaro}
\begin{itemize}
\item Come mai il dollaro \keystrokebftt{\$} \`e un simbolo speciale?
Lo si usa per separare l'ambiente matematico dal testo.\\[1ex]
\begin{exampletwouptiny}
% senza ambiente matematico:
Siano a e b due interi positivi
diversi, e sia c = a - b + 1.

% molto meglio, no?
Siano $a$ e $b$ due interi positivi
diversi, e sia $c = a - b + 1$.
\end{exampletwouptiny}
\item I simboli dollaro vanno sempre usati in coppia --- un per aprire l'ambiente
matematico, l'altro per chiuderlo.
\item Come al solito, \LaTeX{} gestisce la spaziatura in automatico,
ignorando quella dell'utente.
\begin{exampletwouptiny}
Sia $y=mx+b$ dove \ldots

Sia $y = m x + b$ dove \ldots
\end{exampletwouptiny}
\end{itemize}
\end{frame}

%%%%%%%%%%%%%%%%%%%%%%%%%%%%%%%%%%%%%%%%%%%%%%%%%%%%%%%%%%%%%%%%%%%%%%%%%%%%%%%
%%%%%%%%%%%%%%%%%%%%%%%%%%%%%%%%%%%%%%%%%%%%%%%%%%%%%%%%%%%%%%%%%%%%%%%%%%%%%%%
%%%%%%%%%%%%%%%%%%%%%%%%%%%%%%%%%%%%%%%%%%%%%%%%%%%%%%%%%%%%%%%%%%%%%%%%%%%%%%%
\begin{frame}[fragile]{\insertsubsection{}: Notazione}
\begin{itemize}
\item Usate circonflesso \keystrokebftt{\^} per gli apici e \emph{underscore} \keystrokebftt{\_} per i pedici.

\begin{exampletwouptiny}
$y = c_2 x^2 + c_1 x + c_0$
\end{exampletwouptiny}
\vskip 2ex

\item Usate le parentesi graffe \keystrokebftt{\{} \keystrokebftt{\}} per apici e 
pedici pi\`u lunghi.
\begin{exampletwouptiny}
$F_n = F_n-1 + F_n-2$     % oops!

$F_n = F_{n-1} + F_{n-2}$ % ok!
\end{exampletwouptiny}
\vskip 2ex

\item \LaTeX{} offre molti comandi per rappresentare le lettere greche e la notazione pi\`u comune.

\begin{exampletwouptiny}
$\mu = \alpha \int_t e^{q/rt} dt$

$\Omega = \sum_{k=1}^{n} \omega_k$
\end{exampletwouptiny}
\end{itemize}
\end{frame}

%%%%%%%%%%%%%%%%%%%%%%%%%%%%%%%%%%%%%%%%%%%%%%%%%%%%%%%%%%%%%%%%%%%%%%%%%%%%%%%
%%%%%%%%%%%%%%%%%%%%%%%%%%%%%%%%%%%%%%%%%%%%%%%%%%%%%%%%%%%%%%%%%%%%%%%%%%%%%%%
%%%%%%%%%%%%%%%%%%%%%%%%%%%%%%%%%%%%%%%%%%%%%%%%%%%%%%%%%%%%%%%%%%%%%%%%%%%%%%%
\begin{frame}[fragile]{\insertsubsection{}: Equazioni non in linea}
\begin{itemize}
\item Un'equazione lunga e complessa, andrebbe mostrata a parte usando
\cmdbegin{equation} e \cmdend{equation}.\\[2ex]
\begin{exampletwouptiny}
Le radici di un'equazione quadratica
sono date da
\begin{equation}
x = \frac{-b \pm \sqrt{b^2 - 4ac}}
         {2a}
\end{equation}
dove $a$, $b$ e $c$ sono \ldots
\end{exampletwouptiny}
\vskip 1em
{\scriptsize Attenzione: \LaTeX{} ignora gli spazi in ambiente matematico, ma non
\`e in grado di gestire le line vuote --- non usatele!}
\end{itemize}
\end{frame}

%%%%%%%%%%%%%%%%%%%%%%%%%%%%%%%%%%%%%%%%%%%%%%%%%%%%%%%%%%%%%%%%%%%%%%%%%%%%%%%
%%%%%%%%%%%%%%%%%%%%%%%%%%%%%%%%%%%%%%%%%%%%%%%%%%%%%%%%%%%%%%%%%%%%%%%%%%%%%%%
%%%%%%%%%%%%%%%%%%%%%%%%%%%%%%%%%%%%%%%%%%%%%%%%%%%%%%%%%%%%%%%%%%%%%%%%%%%%%%%
\begin{frame}[fragile]{Intermezzo: Ambienti}
\begin{itemize}
\item \bftt{equation} \`e un \emph{ambiente} --- un \structure{contesto sematico}.
\item Un comando pu\`o produrre risultati diversi in contesti diversi.
\begin{exampletwouptiny}
Possiamo scrivere
$ \Omega = \sum_{k=1}^{n} \omega_k $
in linea, o usare un ambiente
\begin{equation}
  \Omega = \sum_{k=1}^{n} \omega_k
\end{equation}
per mostrarlo.
\end{exampletwouptiny}
\vskip 2ex
\item Notate che $\Sigma$ \`e pi\`u grande nell'ambiente \bftt{equation}, e
che apici e pedici cambiano di posizione, nonostante si siano usati gli
stessi comandi
\vskip 1em
{\scriptsize Per simmetria, avremmo potuto scrivere \bftt{\$...\$} come
\cmdbegin{math}\bftt{...}\cmdend{math}.}
\end{itemize}
\end{frame}

%%%%%%%%%%%%%%%%%%%%%%%%%%%%%%%%%%%%%%%%%%%%%%%%%%%%%%%%%%%%%%%%%%%%%%%%%%%%%%%
%%%%%%%%%%%%%%%%%%%%%%%%%%%%%%%%%%%%%%%%%%%%%%%%%%%%%%%%%%%%%%%%%%%%%%%%%%%%%%%
%%%%%%%%%%%%%%%%%%%%%%%%%%%%%%%%%%%%%%%%%%%%%%%%%%%%%%%%%%%%%%%%%%%%%%%%%%%%%%%
\begin{frame}[fragile]{Intermezzo: Ambienti}
\begin{itemize}
\item I comandi \cmdbs{begin} e \cmdbs{end} si possono usare per creare molti
ambienti diversi.
\vskip 2ex

\item Gli ambienti \bftt{itemize} ed \bftt{enumerate} generano liste.
\begin{exampletwouptiny}
\begin{itemize} % lista puntata
\item Biscotti
\item T\'e
\end{itemize}

\begin{enumerate} % lista numerata
\item Biscotti
\item T\'e
\end{enumerate}
\end{exampletwouptiny}
\end{itemize}
\end{frame}

%%%%%%%%%%%%%%%%%%%%%%%%%%%%%%%%%%%%%%%%%%%%%%%%%%%%%%%%%%%%%%%%%%%%%%%%%%%%%%%
%%%%%%%%%%%%%%%%%%%%%%%%%%%%%%%%%%%%%%%%%%%%%%%%%%%%%%%%%%%%%%%%%%%%%%%%%%%%%%%
%%%%%%%%%%%%%%%%%%%%%%%%%%%%%%%%%%%%%%%%%%%%%%%%%%%%%%%%%%%%%%%%%%%%%%%%%%%%%%%
\begin{frame}[fragile]{Intermezzo: Pacchetti}

\begin{itemize}
\item Tutti i comandi e gli ambienti mostrati fino ad adesso, sono parte di
\LaTeX{} \emph{base}.

\item I \emph{pacchetti} sono librerie di comandi e ambienti aggiuntivi:
ci sono migliaia di pacchetti liberamente disponibili.

\item I pacchetti che vogliamo usare vanno caricati esplicitamente usando il comando
\cmdbs{usepackage} nel \structure{preambolo}.

\item Esempio: \bftt{amsmath} della American Mathematical Society.
\begin{minted}[fontsize=\small,frame=single]{latex}
\documentclass{article}
\usepackage{amsmath} % preambolo
\begin{document}
% da ora in poi possiamo usare i comandi di amsmath...
\end{document}
\end{minted}
\end{itemize}
\end{frame}

%%%%%%%%%%%%%%%%%%%%%%%%%%%%%%%%%%%%%%%%%%%%%%%%%%%%%%%%%%%%%%%%%%%%%%%%%%%%%%%
%%%%%%%%%%%%%%%%%%%%%%%%%%%%%%%%%%%%%%%%%%%%%%%%%%%%%%%%%%%%%%%%%%%%%%%%%%%%%%%
%%%%%%%%%%%%%%%%%%%%%%%%%%%%%%%%%%%%%%%%%%%%%%%%%%%%%%%%%%%%%%%%%%%%%%%%%%%%%%%
\begin{frame}[fragile]{\insertsubsection{}: Esempi con \bftt{amsmath}}
\begin{itemize}
\item Usa \bftt{equation*} per inserire equazioni non numerate.
\begin{exampletwouptiny}
\begin{equation*}
  \Omega = \sum_{k=1}^{n} \omega_k
\end{equation*}
\end{exampletwouptiny}
\item \LaTeX{} tratta lettere adiacenti come variabili moltiplicate tra di loro, ma non \`e sempre desiderabile: \bftt{amsmath} definisce comandi per la maggior parte delle funzioni matematiche.
\begin{exampletwouptiny}
\begin{equation*} % sbagliato!
 min_{x,y} (1-x)^2 + 100(y-x^2)^2
\end{equation*}
\begin{equation*} % giusto!
\min_{x,y}{(1-x)^2 + 100(y-x^2)^2}
\end{equation*}
\end{exampletwouptiny}
\item Per quelle non predefinite, si usa \cmdbs{operatorname}.
\begin{exampletwouptiny}
\begin{equation*}
\beta_i =
\frac{\operatorname{Cov}(R_i, R_m)}
     {\operatorname{Var}(R_m)}
\end{equation*}
\end{exampletwouptiny}
\end{itemize}
\end{frame}

%%%%%%%%%%%%%%%%%%%%%%%%%%%%%%%%%%%%%%%%%%%%%%%%%%%%%%%%%%%%%%%%%%%%%%%%%%%%%%%
%%%%%%%%%%%%%%%%%%%%%%%%%%%%%%%%%%%%%%%%%%%%%%%%%%%%%%%%%%%%%%%%%%%%%%%%%%%%%%%
%%%%%%%%%%%%%%%%%%%%%%%%%%%%%%%%%%%%%%%%%%%%%%%%%%%%%%%%%%%%%%%%%%%%%%%%%%%%%%%
\begin{frame}[fragile]{\insertsubsection{}: Esempi con \bftt{amsmath}}
\begin{itemize}{\small
\item Allinea una sequenza di equazioni con il simbolo di uguale
\begin{align*}
(x+1)^3 &= (x+1)(x+1)(x+1) \\
        &= (x+1)(x^2 + 2x + 1) \\
        &= x^3 + 3x^2 + 3x + 1
\end{align*}
utilizzando l'ambiente \bftt{align*}.

% for whatever reason, this doesn't play well with the twoup environment
\begin{minted}[fontsize=\small,frame=single]{latex}
\begin{align*}
(x+1)^3 &= (x+1)(x+1)(x+1) \\
        &= (x+1)(x^2 + 2x + 1) \\
        &= x^3 + 3x^2 + 3x + 1
\end{align*}
\end{minted}
\item Una ampersand \keystrokebftt{\&} separa la colonna sinistra (prima di
$=$) dalla colonna destra (dopo di $=$).
\item Un doppio backslash \keystrokebftt{\bs}\keystrokebftt{\bs} inizia una
nuova linea.
}\end{itemize}
\end{frame}


%%%%%%%%%%%%%%%%%%%%%%%%%%%%%%%%%%%%%%%%%%%%%%%%%%%%%%%%%%%%%%%%%%%%%%%%%%%%%%%
%%%%%%%%%%%%%%%%%%%%%%%%%%%%%%%%%%%%%%%%%%%%%%%%%%%%%%%%%%%%%%%%%%%%%%%%%%%%%%%
%%%%%%%%%%%%%%%%%%%%%%%%%%%%%%%%%%%%%%%%%%%%%%%%%%%%%%%%%%%%%%%%%%%%%%%%%%%%%%%
\begin{frame}[fragile]{Esercizio 2}

\begin{block}{Scrivi questo in \LaTeX:}
Siano $X_1, X_2, \ldots, X_n$ una serie di variabili casuali indipendenti ed
identicamente distribuite tali per cui $\operatorname{E}[X_i] = \mu$ e
$\operatorname{Var}[X_i] = \sigma^2 < \infty$, con media
\begin{equation*}
S_n = \frac{1}{n}\sum_{i}^{n} X_i
\end{equation*}
Per $n$ che tende ad infinito, le variabili casuali
$\sqrt{n}(S_n - \mu)$ convergono in senso distribuzionale ad una gaussiana $N(0, \sigma^2)$.
\end{block}
\vskip 2ex
\begin{center}
\fbox{\href{\wlnewdoc{basics-exercise-2.tex}}{%
Clicca per aprire questo esercizio in \wllogo{}}}
\end{center}
\begin{itemize}
\item Suggerimento: il comando per $\infty$ \`e \cmdbs{infty}.
\item Qui potete trovare la 
\fbox{\href{\wlnewdoc{basics-exercise-2-solution.tex}}{%
mia soluzione}}.
\end{itemize}
\end{frame}

%%%%%%%%%%%%%%%%%%%%%%%%%%%%%%%%%%%%%%%%%%%%%%%%%%%%%%%%%%%%%%%%%%%%%%%%%%%%%%%
%%%%%%%%%%%%%%%%%%%%%%%%%%%%%%%%%%%%%%%%%%%%%%%%%%%%%%%%%%%%%%%%%%%%%%%%%%%%%%%
%%%%%%%%%%%%%%%%%%%%%%%%%%%%%%%%%%%%%%%%%%%%%%%%%%%%%%%%%%%%%%%%%%%%%%%%%%%%%%%
\begin{frame}{Fine della prima parte}
\begin{itemize}
\item \structure{Congratulazioni!} Avete imparato a\ldots
\begin{itemize}
\item Inserire testo in \LaTeX.
\item Utilizzare i comandi di base.
\item Gestire gli errori via via che compaiono.
\item Scrivere della bellissima matematica.
\item Utilizzare alcuni ambienti.
\item Caricare pacchetti.
\end{itemize}
\item Non \`e fantastico?
\item Nella \structure{seconda parte}, impareremo ad usare \LaTeX{} per scrivere
documenti strutturati con sezioni, riferimenti incrociati, figure,
tabelle, bibliografia\ldots. Alla prossima!
\end{itemize}
\end{frame}

\end{document}
