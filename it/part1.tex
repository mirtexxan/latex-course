\documentclass{beamer}

\input{preamble.tex}

\subtitle{Parte 1: Le basi}

\begin{document}

%%%%%%%%%%%%%%%%%%%%%%%%%%%%%%%%%%%%%%%%%%%%%%%%%%%%%%%%%%%%%%%%%%%%%%%%%%%%%%%
%%%%%%%%%%%%%%%%%%%%%%%%%%%%%%%%%%%%%%%%%%%%%%%%%%%%%%%%%%%%%%%%%%%%%%%%%%%%%%%
%%%%%%%%%%%%%%%%%%%%%%%%%%%%%%%%%%%%%%%%%%%%%%%%%%%%%%%%%%%%%%%%%%%%%%%%%%%%%%%
\begin{frame}
\titlepage
\end{frame}

%%%%%%%%%%%%%%%%%%%%%%%%%%%%%%%%%%%%%%%%%%%%%%%%%%%%%%%%%%%%%%%%%%%%%%%%%%%%%%%
%%%%%%%%%%%%%%%%%%%%%%%%%%%%%%%%%%%%%%%%%%%%%%%%%%%%%%%%%%%%%%%%%%%%%%%%%%%%%%%
%%%%%%%%%%%%%%%%%%%%%%%%%%%%%%%%%%%%%%%%%%%%%%%%%%%%%%%%%%%%%%%%%%%%%%%%%%%%%%%
\begin{frame}{Esempi}
\begin{figure}
\includegraphics[height=0.7\textheight]{esempio1}
\end{figure}
\end{frame}

\begin{frame}{Esempi}
\shapepar{\heartshape}
If I speak in the tongues of men or of angels, but do not have love, I am only a resounding gong or a clanging cymbal. If I have the gift of prophecy and can fathom all mysteries and all knowledge, and if I have a faith that can move mountains, but do not have love, I am nothing.  If I give all I possess to the poor and give over my body to hardship that I may boast, but do not have love, I gain nothing. Love is patient, love is kind. It does not envy, it does not boast, it is not proud.  It does not dishonor others, it is not self-seeking, it is not easily angered, it keeps no record of wrongs.  Love does not delight in evil but rejoices with the truth. It always protects, always trusts, always hopes, always perseveres.
\end{frame}

\begin{frame}{Esempi}
\bigskip\bigskip\bigskip
\foreignlanguage{polutonikogreek}{
\begin{verse}Th| p'anta dido'ush| ka`i >apolambano'ush| f'usei <o pepaideum'enos ka`i a>id'hmwn l'egei; <<d`os, <`o j'eleis, >ap'olabe, <`o j'eleic>>.
L'egei d`e to~uto o>u katajrasun'omenos, >all`a peijarq~wn m'onon ka`i e>uno~wn a>ut~h|.
\end{verse}}
\begin{flushright}
- Marco Aurelio, \emph{Ricordi}
\end{flushright}
\end{frame}

\begin{frame}{Esempi}
\begin{figure}
\includegraphics[width=0.9\textwidth]{esempio4}
\end{figure}
\end{frame}

%%%%%%%%%%%%%%%%%%%%%%%%%%%%%%%%%%%%%%%%%%%%%%%%%%%%%%%%%%%%%%%%%%%%%%%%%%%%%%%
%%%%%%%%%%%%%%%%%%%%%%%%%%%%%%%%%%%%%%%%%%%%%%%%%%%%%%%%%%%%%%%%%%%%%%%%%%%%%%%
%%%%%%%%%%%%%%%%%%%%%%%%%%%%%%%%%%%%%%%%%%%%%%%%%%%%%%%%%%%%%%%%%%%%%%%%%%%%%%%
\begin{frame}{Perch\'e \LaTeX{}?}
\begin{itemize}
\item Permette di realizzare documenti professionali e esteticamente appaganti
\begin{itemize}
\item Specialmente se contengono \structure{matematica}
\end{itemize}
%
\item \`E stato creato da scienziati, per scienziati
\begin{itemize}
\item Una comunit\`a enorme e molto attiva
\item \ldots ma il suo uso \`e estendibile ad ogni campo
\end{itemize}
%
\item \`E potentissimo --- ed estendibile a piacimento
\begin{itemize}
\item \structure{Pacchetti} per articoli scientifici, libri, presentazioni,
fogli di calcolo, \ldots
\end{itemize}
\end{itemize}
\end{frame}

%%%%%%%%%%%%%%%%%%%%%%%%%%%%%%%%%%%%%%%%%%%%%%%%%%%%%%%%%%%%%%%%%%%%%%%%%%%%%%%
%%%%%%%%%%%%%%%%%%%%%%%%%%%%%%%%%%%%%%%%%%%%%%%%%%%%%%%%%%%%%%%%%%%%%%%%%%%%%%%
%%%%%%%%%%%%%%%%%%%%%%%%%%%%%%%%%%%%%%%%%%%%%%%%%%%%%%%%%%%%%%%%%%%%%%%%%%%%%%%
\begin{frame}{Come si pronuncia?}

\begin{itemize}
\item \TeX{} \`e stato creato a Stanford da Donald E. Knuth
\item Il nome deriva dalla radice greca di parole come
\textbf{$\tau\epsilon\chi\nu\acute{\eta}$}
che significa \emph{arte} o \emph{tecnica}.
\begin{itemize}
\item la pronuncia dovrebbe essere \emph{tech} (come il tedesco \emph{Bach})
\item in italiano solitamente si usa \emph{tek}
\end{itemize}
\item \LaTeX{} \`e un set di macro per \TeX creato da Leslie Lamport

\begin{itemize}
\item oramai nessuno usa pi\`u \TeX base
\item la pronuncia usuale italiana \`e \emph{latek}.
\end{itemize}
\end{itemize}
\end{frame}

%%%%%%%%%%%%%%%%%%%%%%%%%%%%%%%%%%%%%%%%%%%%%%%%%%%%%%%%%%%%%%%%%%%%%%%%%%%%%%%
%%%%%%%%%%%%%%%%%%%%%%%%%%%%%%%%%%%%%%%%%%%%%%%%%%%%%%%%%%%%%%%%%%%%%%%%%%%%%%%
%%%%%%%%%%%%%%%%%%%%%%%%%%%%%%%%%%%%%%%%%%%%%%%%%%%%%%%%%%%%%%%%%%%%%%%%%%%%%%%
%\begin{frame}{Dove funziona?}
%
%\begin{itemize}
%\item Non dovrebbe stupire che per usare LATEX occorra un calcolatore.
%\item Quale? Non importa, basta che vada!
%\item Il sistema si può compilare su tutte le più diffuse architetture e
%i più diffusi sistemi operativi.
%\end{itemize}
%\begin{block}<2->{Bastano?}
%
%\begin{itemize}
%\item GNU/Linux, FreeBSD, NetBSD
%\item SunOS$^{\lyxmathsym{\texttrademark}}$
%\item Apple$^{\circledR}$ Mac OS X (e versioni precedenti)
%\item Microsoft$^{\circledR}$ Windows$^{\lyxmathsym{\texttrademark}}$
%\end{itemize}
%\end{block}
%\end{frame}

%%%%%%%%%%%%%%%%%%%%%%%%%%%%%%%%%%%%%%%%%%%%%%%%%%%%%%%%%%%%%%%%%%%%%%%%%%%%%%%
%%%%%%%%%%%%%%%%%%%%%%%%%%%%%%%%%%%%%%%%%%%%%%%%%%%%%%%%%%%%%%%%%%%%%%%%%%%%%%%
%%%%%%%%%%%%%%%%%%%%%%%%%%%%%%%%%%%%%%%%%%%%%%%%%%%%%%%%%%%%%%%%%%%%%%%%%%%%%%%
%\begin{frame}{Distribuzioni}
%
%\begin{block}{Cosa scaricare}
%
%\begin{itemize}
%\item GNU/Linux: te\TeX{} come sorgente o come pacchetto;
%\item Apple Mac OS X: Mac\TeX;
%\item Microsoft Windows: Mik\TeX{} (ce ne sono anche altri)
%\end{itemize}
%\end{block}
%\begin{itemize}
%\item I dettagli sono lasciati al lettore come \alert{facile} esercizio.
%\item Consiglio: Google è vostro amico...
%\item Ottenuta la distribuzione, il codice si crive con un editor di testi
%
%\begin{itemize}
%\item Un qualsiasi blocco note serve allo scopo
%\item Esistono anche ambienti integrati (e.g. \TeX works)
%\end{itemize}
%\end{itemize}
%\end{frame}

%%%%%%%%%%%%%%%%%%%%%%%%%%%%%%%%%%%%%%%%%%%%%%%%%%%%%%%%%%%%%%%%%%%%%%%%%%%%%%%
%%%%%%%%%%%%%%%%%%%%%%%%%%%%%%%%%%%%%%%%%%%%%%%%%%%%%%%%%%%%%%%%%%%%%%%%%%%%%%%
%%%%%%%%%%%%%%%%%%%%%%%%%%%%%%%%%%%%%%%%%%%%%%%%%%%%%%%%%%%%%%%%%%%%%%%%%%%%%%%
%\begin{frame}[allowframebreaks]{Pregi e difetti di \LaTeX}
%
%\begin{block}{Pregi}
%
%\begin{itemize}
%\item Nelle distribuzioni sono compresi stili per la composizione di articoli
%o \alert{documenti di livello professionale}.
%\item La composizione di \alert{formule matematiche} è di altissimo livello.
%\item L\textquoteright utente deve concentrarsi solo sul \alert{contenuto del documento e non sulla sua forma finale}.
%\item Si possono generare note a piè di pagina, riferimenti incrociati,
%bibliografie e indici in modo \alert{automatico}.
%\item Migliaia di persone usano \LaTeX: è facile trovare supporto;
%\item È gratis.
%\end{itemize}
%\end{block}
%
%\begin{block}{Difetti}
%
%\begin{itemize}
%\item Non permette di \alert{vedere il documento finale} mentre lo si scrive.
%\item Richiede di \alert{intercalare il testo con comandi di composizione}
%che rispecchiano la struttura logica del testo.
%\item Non è un ambiente integrato.
%\item Non dà denaro alle case di software.
%\end{itemize}
%\end{block}
%\end{frame}

%%%%%%%%%%%%%%%%%%%%%%%%%%%%%%%%%%%%%%%%%%%%%%%%%%%%%%%%%%%%%%%%%%%%%%%%%%%%%%%
%%%%%%%%%%%%%%%%%%%%%%%%%%%%%%%%%%%%%%%%%%%%%%%%%%%%%%%%%%%%%%%%%%%%%%%%%%%%%%%
%%%%%%%%%%%%%%%%%%%%%%%%%%%%%%%%%%%%%%%%%%%%%%%%%%%%%%%%%%%%%%%%%%%%%%%%%%%%%%%
\begin{frame}[fragile]{Come funziona \LaTeX?}

Facciamo un paragone con il signor Pippo che vuole scrivere un programma
nel suo linguaggio preferito, C

\bigskip{}
\begin{enumerate}
\item Pippo scrive un documento di testo che chiamer\`a pippo.c.
\item \label{enu:compila}Pippo dà sul terminale il comando\\

\begin{minted}[frame=single]{bash}
gcc pippo.c
\end{minted}

\item Se la compilazione d\`a errori, Pippo rivede il programma, lo corregge
e ritorna al passo \ref{enu:compila}.
\item Se la compilazione ha successo, viene prodotto un file eseguibile,
\bftt{a.out}.
\item Pippo prova a vedere se il programma fa ciò che desidera.
\end{enumerate}
\end{frame}

%%%%%%%%%%%%%%%%%%%%%%%%%%%%%%%%%%%%%%%%%%%%%%%%%%%%%%%%%%%%%%%%%%%%%%%%%%%%%%%
%%%%%%%%%%%%%%%%%%%%%%%%%%%%%%%%%%%%%%%%%%%%%%%%%%%%%%%%%%%%%%%%%%%%%%%%%%%%%%%
%%%%%%%%%%%%%%%%%%%%%%%%%%%%%%%%%%%%%%%%%%%%%%%%%%%%%%%%%%%%%%%%%%%%%%%%%%%%%%%
\begin{frame}[fragile]{Come funziona \LaTeX?}

Ora Pippo vuole scrivere la documentazione per il suo programma.
\bigskip
\begin{enumerate}
\item Pippo scrive un documento di testo che chiamer\`a \bftt{pippo.tex}.
\item \label{enu:compilalatex}Pippo dà sul terminale il comando

\begin{minted}[frame=single]{bash}
latex pippo.tex
\end{minted}

\item Se la compilazione d\`a errori, Pippo rivede il documento, lo corregge
e ritorna al passo \ref{enu:compilalatex}.
\item Se la compilazione ha successo, viene prodotto un file \bftt{pippo.dvi}.
\item Pippo chiama il visualizzatore con

\begin{minted}[frame=single]{bash}
xdvi pippo
\end{minted}

e controlla che non ci siano errori concettuali.
\end{enumerate}
\end{frame}

%%%%%%%%%%%%%%%%%%%%%%%%%%%%%%%%%%%%%%%%%%%%%%%%%%%%%%%%%%%%%%%%%%%%%%%%%%%%%%%
%%%%%%%%%%%%%%%%%%%%%%%%%%%%%%%%%%%%%%%%%%%%%%%%%%%%%%%%%%%%%%%%%%%%%%%%%%%%%%%
%%%%%%%%%%%%%%%%%%%%%%%%%%%%%%%%%%%%%%%%%%%%%%%%%%%%%%%%%%%%%%%%%%%%%%%%%%%%%%%
\begin{frame}[fragile]{Come funziona pdf\LaTeX?}

Ora Pippo vuole scrivere la documentazione per il suo programma in
un formato pi\`u comune.
\bigskip
\begin{enumerate}
\item Pippo ha gi\`a il documento di testo chiamato \bftt{pippo.tex}.
\item Pippo d\`a sul terminale il comando
\begin{minted}[frame=single]{bash}
pdflatex pippo.tex
\end{minted}
\item Viene prodotto un file \bftt{pippo.pdf}.
\end{enumerate}
\bigskip
Dato che il documento \bftt{pippo.tex} \`e lo stesso di prima,
il documento finale \`e solo la resa in PDF di quello precedente.
\end{frame}

%%%%%%%%%%%%%%%%%%%%%%%%%%%%%%%%%%%%%%%%%%%%%%%%%%%%%%%%%%%%%%%%%%%%%%%%%%%%%%%
%%%%%%%%%%%%%%%%%%%%%%%%%%%%%%%%%%%%%%%%%%%%%%%%%%%%%%%%%%%%%%%%%%%%%%%%%%%%%%%
%%%%%%%%%%%%%%%%%%%%%%%%%%%%%%%%%%%%%%%%%%%%%%%%%%%%%%%%%%%%%%%%%%%%%%%%%%%%%%%
\begin{frame}[fragile]{Come funziona?}
\begin{itemize}
\item Un documento \LaTeX \`e composto di \texttt{testo semplice} inframezzato a
\cmd{comandi} che ne descrivono la struttura e il significato.
\item L'applicazione \texttt{latex} compila il testo e i comandi per produrre un documento perfettamente formattato.
\end{itemize}
\vskip 2ex
\begin{center}
\begin{minted}[frame=single]{latex}
La rana in Spagna \emph{gracida} in campagna.
\end{minted}
\vskip 2ex
\tikz\node[single arrow,fill=gray,font=\ttfamily\bfseries,%
  rotate=270,xshift=-1em]{latex};
\vskip 2ex
\fbox{La rana in Spagna \emph{gracida} in campagna.}
\end{center}
\end{frame}

%%%%%%%%%%%%%%%%%%%%%%%%%%%%%%%%%%%%%%%%%%%%%%%%%%%%%%%%%%%%%%%%%%%%%%%%%%%%%%%
%%%%%%%%%%%%%%%%%%%%%%%%%%%%%%%%%%%%%%%%%%%%%%%%%%%%%%%%%%%%%%%%%%%%%%%%%%%%%%%
%%%%%%%%%%%%%%%%%%%%%%%%%%%%%%%%%%%%%%%%%%%%%%%%%%%%%%%%%%%%%%%%%%%%%%%%%%%%%%%
\begin{frame}[fragile]{Alcuni esempi di comandi\ldots}
\begin{exampletwoup}
\begin{itemize}
\item T\'e
\item Latte
\item Biscotti
\end{itemize}
\end{exampletwoup}
\vskip 2ex
\begin{exampletwoup}
\begin{figure}
\includegraphics{pulcino}
\end{figure}
\end{exampletwoup}
\vskip 2ex
\begin{exampletwoup}
\begin{equation}
\alpha + \beta + 1
\end{equation}
\end{exampletwoup}

\tiny{Immagine tratta da \url{http://www.andy-roberts.net/writing/latex/importing_images}}
\end{frame}

%%%%%%%%%%%%%%%%%%%%%%%%%%%%%%%%%%%%%%%%%%%%%%%%%%%%%%%%%%%%%%%%%%%%%%%%%%%%%%%
%%%%%%%%%%%%%%%%%%%%%%%%%%%%%%%%%%%%%%%%%%%%%%%%%%%%%%%%%%%%%%%%%%%%%%%%%%%%%%%
%%%%%%%%%%%%%%%%%%%%%%%%%%%%%%%%%%%%%%%%%%%%%%%%%%%%%%%%%%%%%%%%%%%%%%%%%%%%%%%
\begin{frame}[fragile]{Un vero e proprio cambio di \structure{paradigma}}

\begin{itemize}
\item Usate i comandi per descrivere `cio che \`e' e non `ci\`o che appare'
\item Concentratevi sul contenuto
\item Lasciate fare a \LaTeX{} il suo lavoro!
\end{itemize}
\end{frame}

%%%%%%%%%%%%%%%%%%%%%%%%%%%%%%%%%%%%%%%%%%%%%%%%%%%%%%%%%%%%%%%%%%%%%%%%%%%%%%%
%%%%%%%%%%%%%%%%%%%%%%%%%%%%%%%%%%%%%%%%%%%%%%%%%%%%%%%%%%%%%%%%%%%%%%%%%%%%%%%
%%%%%%%%%%%%%%%%%%%%%%%%%%%%%%%%%%%%%%%%%%%%%%%%%%%%%%%%%%%%%%%%%%%%%%%%%%%%%%%
\section{Le basi}

%%%%%%%%%%%%%%%%%%%%%%%%%%%%%%%%%%%%%%%%%%%%%%%%%%%%%%%%%%%%%%%%%%%%%%%%%%%%%%%
%%%%%%%%%%%%%%%%%%%%%%%%%%%%%%%%%%%%%%%%%%%%%%%%%%%%%%%%%%%%%%%%%%%%%%%%%%%%%%%
%%%%%%%%%%%%%%%%%%%%%%%%%%%%%%%%%%%%%%%%%%%%%%%%%%%%%%%%%%%%%%%%%%%%%%%%%%%%%%%
\subsection{Iniziamo\ldots}
\begin{frame}[fragile]{\insertsubsection}
\begin{itemize}
\item Il documento \LaTeX{} minimale:
\inputminted[frame=single]{latex}{basics.tex}
\item Tutti i comandi iniziano con un \emph{backslash} \keystrokebftt{\bs}.
\item Ogni documenti inizia con un comando \cmdbs{documentclass}.
\item L'\emph{argomento} tra parentesi graffe \keystrokebftt{\{} \keystrokebftt{\}} indica a \LaTeX{} che tipo di documento stiamo creano: un \bftt{article}.
\item Il simbolo di percento \keystrokebftt{\%} d\`a inizio ad un \emph{commento}
--- \LaTeX{} ignorer\`a il resto della riga.
\end{itemize}
\end{frame}

%%%%%%%%%%%%%%%%%%%%%%%%%%%%%%%%%%%%%%%%%%%%%%%%%%%%%%%%%%%%%%%%%%%%%%%%%%%%%%%
%%%%%%%%%%%%%%%%%%%%%%%%%%%%%%%%%%%%%%%%%%%%%%%%%%%%%%%%%%%%%%%%%%%%%%%%%%%%%%%
%%%%%%%%%%%%%%%%%%%%%%%%%%%%%%%%%%%%%%%%%%%%%%%%%%%%%%%%%%%%%%%%%%%%%%%%%%%%%%%
\begin{frame}[fragile]{\insertsubsection{} con \wllogo}
\begin{itemize}
\item Overleaf \`e un webapp per scrivere documenti in \LaTeX.
\item `Compila' un sorgente \LaTeX{} e mostra i risultati in automatico e in
tempo reale.
\vskip 2em
\begin{center}
\fbox{\href{\wlnewdoc{basics.tex}}{%
Clicca qui per aprire il documento minimale di prima con \wllogo{}}}
\\[1ex]\scriptsize{}
Per migliore compatibilit\`a, consiglio di usare \href{http://www.google.com/chrome}{Chrome} o una versione recente di \href{http://www.mozilla.org/en-US/firefox/new/}{FireFox}.
\end{center}
\vskip 2ex
\item Nel resto del corso, provate ad eseguire gli esempi, copiandoli
direttamente su Overleaf.
\item \textbf{Fatelo davvero, \`e il miglior modo di imparare!}
\end{itemize}
\end{frame}

%%%%%%%%%%%%%%%%%%%%%%%%%%%%%%%%%%%%%%%%%%%%%%%%%%%%%%%%%%%%%%%%%%%%%%%%%%%%%%%
%%%%%%%%%%%%%%%%%%%%%%%%%%%%%%%%%%%%%%%%%%%%%%%%%%%%%%%%%%%%%%%%%%%%%%%%%%%%%%%
%%%%%%%%%%%%%%%%%%%%%%%%%%%%%%%%%%%%%%%%%%%%%%%%%%%%%%%%%%%%%%%%%%%%%%%%%%%%%%%
\subsection{Inserire il testo}
\begin{frame}[fragile]{\insertsubsection{}}
\small
\begin{itemize}
\item Tutto il testo di un qualunque documento va inserito tra \cmdbegin{document}
e \cmdend{document}.
\item Nella maggior parte dei casi, potete inserire testo normalmente.
\begin{exampletwouptiny}
Le parole sono separate da uno o pi\`u
spazi.

I paragrafi sono separati da una o pi\`u
righe vuote.
\end{exampletwouptiny}
\item Lo spazio nel file sorgente viene aggregato nell'output.
\begin{exampletwouptiny}
La   rana       in Spagna
gracida in		 montagna.
\end{exampletwouptiny}
\end{itemize}
\end{frame}

%%%%%%%%%%%%%%%%%%%%%%%%%%%%%%%%%%%%%%%%%%%%%%%%%%%%%%%%%%%%%%%%%%%%%%%%%%%%%%%
%%%%%%%%%%%%%%%%%%%%%%%%%%%%%%%%%%%%%%%%%%%%%%%%%%%%%%%%%%%%%%%%%%%%%%%%%%%%%%%
%%%%%%%%%%%%%%%%%%%%%%%%%%%%%%%%%%%%%%%%%%%%%%%%%%%%%%%%%%%%%%%%%%%%%%%%%%%%%%%
\begin{frame}[fragile]{\insertsubsection{}: Caveats}
\small
\begin{itemize}
\item Le virgolette possono essere un po\' tricky: va usato un apostrofo rovesciato \keystroke{\`{}} a sinistra e un apostrofo semplice \keystroke{\'{}} a destra.
\begin{exampletwouptiny}
Virgolette semplici: `testo'.

Virgolette doppie: ``testo''.
\end{exampletwouptiny}

\item Alcuni simboli comuni hanno significato speciale in \LaTeX:\\[1ex]
\begin{tabular}{cl}
\keystrokebftt{\%} & percento					\\
\keystrokebftt{\#} & cancelleto / tag			\\
\keystrokebftt{\&} & e commerciale / ampersand	\\
\keystrokebftt{\$} & dollaro					\\
\end{tabular}
\item Se provate ad inserirli direttamente, otterrete un messaggio di errore.
Se volete mostrarli nel documento dovete fare \emph{escape}, facendoli precedere
da un backslash.
\begin{exampletwoup}
\$\%\&\#!
\end{exampletwoup}
\end{itemize}
\end{frame}

%%%%%%%%%%%%%%%%%%%%%%%%%%%%%%%%%%%%%%%%%%%%%%%%%%%%%%%%%%%%%%%%%%%%%%%%%%%%%%%
%%%%%%%%%%%%%%%%%%%%%%%%%%%%%%%%%%%%%%%%%%%%%%%%%%%%%%%%%%%%%%%%%%%%%%%%%%%%%%%
%%%%%%%%%%%%%%%%%%%%%%%%%%%%%%%%%%%%%%%%%%%%%%%%%%%%%%%%%%%%%%%%%%%%%%%%%%%%%%%
\begin{frame}[fragile]{Gestire gli errori}
\begin{itemize}
\item \LaTeX{} si pu\'o confondere quando cerca di compilare un documento.
Se succede, si interrompe con un messaggio di errore, che dovrete correggere,
se volete avere qualche speranza di produrre un documento.
\item Per esempio, se provate a scrivere \cmdbs{epmh} invece di \cmdbs{emph},
\LaTeX{} si lamenter\`a con un errore ``undefined control sequence'' dato che, effettivamente, ``epmh'' non esiste come comando.
\end{itemize}
\begin{block}{Qualche consiglio sugli errori}
\begin{enumerate}
\item Niente panico! Succede a tutti.
\item Correggeli immediatamente --- se quello che avete appena scritto ha causato
un errore, perlomeno sapete da dove partire per il debugging.
\item Se ci sono errori multipli, correggeteli uno alla volta iniziando dal primo 
--- potrebbero essere errori a cascata.
\end{enumerate}
\end{block}
\end{frame}

%%%%%%%%%%%%%%%%%%%%%%%%%%%%%%%%%%%%%%%%%%%%%%%%%%%%%%%%%%%%%%%%%%%%%%%%%%%%%%%
%%%%%%%%%%%%%%%%%%%%%%%%%%%%%%%%%%%%%%%%%%%%%%%%%%%%%%%%%%%%%%%%%%%%%%%%%%%%%%%
%%%%%%%%%%%%%%%%%%%%%%%%%%%%%%%%%%%%%%%%%%%%%%%%%%%%%%%%%%%%%%%%%%%%%%%%%%%%%%%
\begin{frame}[fragile]{Esercizio 1}

\begin{block}{Digita questo in \LaTeX:
\footnote{\url{http://en.wikipedia.org/wiki/Economy_of_the_United_States}}}
In March 2006, Congress raised that ceiling an additional \$0.79 trillion to
\$8.97 trillion, which is approximately 68\% of GDP. As of October 4, 2008, the
``Emergency Economic Stabilization Act of 2008'' raised the current debt ceiling
to \$11.3 trillion.
\end{block}
\vskip 2ex
\begin{center}
\fbox{\href{\wlnewdoc{basics-exercise-1.tex}}{%
Clicca per aprire l'esercizio in \wllogo{}}}
\end{center}

\begin{itemize}
\item Suggerimento: occhio ai caratteri speciali!
\item Dopo qualche tentativo,
\fbox{\href{\wlnewdoc{basics-exercise-1-solution.tex}}{%
cliccate qui per la soluzione}}.
\end{itemize}
\end{frame}

%%%%%%%%%%%%%%%%%%%%%%%%%%%%%%%%%%%%%%%%%%%%%%%%%%%%%%%%%%%%%%%%%%%%%%%%%%%%%%%
%%%%%%%%%%%%%%%%%%%%%%%%%%%%%%%%%%%%%%%%%%%%%%%%%%%%%%%%%%%%%%%%%%%%%%%%%%%%%%%
%%%%%%%%%%%%%%%%%%%%%%%%%%%%%%%%%%%%%%%%%%%%%%%%%%%%%%%%%%%%%%%%%%%%%%%%%%%%%%%
\subsection{Ambienti matematici}
\begin{frame}[fragile]{\insertsubsection{}: il dollaro}
\begin{itemize}
\item Come mai il dollaro \keystrokebftt{\$} \`e un simbolo speciale?
Lo si usa per separare l'ambiente matematico dal testo.\\[1ex]
\begin{exampletwouptiny}
% senza ambiente matematico:
Siano a e b due interi positivi
diversi, e sia c = a - b + 1.

% con ambiente matematico:
Siano $a$ e $b$ due interi positivi
diversi, e sia $c = a - b + 1$.
\end{exampletwouptiny}
\item I simboli dollaro vanno sempre usati in coppia --- un per aprire l'ambiente
matematico, l'altro per chiouderlo.
\item Come al solito, \LaTeX{} gestisce la spaziatura in automatico,
ignorando quella dell'utente.
\begin{exampletwouptiny}
Sia $y=mx+b$ dove \ldots

Sia $y = m x + b$ dove \ldots
\end{exampletwouptiny}
\end{itemize}
\end{frame}

%%%%%%%%%%%%%%%%%%%%%%%%%%%%%%%%%%%%%%%%%%%%%%%%%%%%%%%%%%%%%%%%%%%%%%%%%%%%%%%
%%%%%%%%%%%%%%%%%%%%%%%%%%%%%%%%%%%%%%%%%%%%%%%%%%%%%%%%%%%%%%%%%%%%%%%%%%%%%%%
%%%%%%%%%%%%%%%%%%%%%%%%%%%%%%%%%%%%%%%%%%%%%%%%%%%%%%%%%%%%%%%%%%%%%%%%%%%%%%%
\begin{frame}[fragile]{\insertsubsection{}: Notazione}
\begin{itemize}
\item Usa il circonflesso (\emph{caret}) \keystrokebftt{\^} per gli apici e la linea bassa (\emph{underscore}) \keystrokebftt{\_} per i pedici.
\begin{exampletwouptiny}
$y = c_2 x^2 + c_1 x + c_0$
\end{exampletwouptiny}
\vskip 2ex

\item Usa le parentesi graffe \keystrokebftt{\{} \keystrokebftt{\}} per apici e 
pedici compositi.
\begin{exampletwouptiny}
$F_n = F_n-1 + F_n-2$     % oops!

$F_n = F_{n-1} + F_{n-2}$ % ok!
\end{exampletwouptiny}
\vskip 2ex

\item Ci sono comandi per le lettere Greche e la notazione pi\'u comune.
\begin{exampletwouptiny}
$\mu = \alpha \int_t e^{q/rt} dt$

$\Omega = \sum_{k=1}^{n} \omega_k$
\end{exampletwouptiny}
\end{itemize}
\end{frame}

%%%%%%%%%%%%%%%%%%%%%%%%%%%%%%%%%%%%%%%%%%%%%%%%%%%%%%%%%%%%%%%%%%%%%%%%%%%%%%%
%%%%%%%%%%%%%%%%%%%%%%%%%%%%%%%%%%%%%%%%%%%%%%%%%%%%%%%%%%%%%%%%%%%%%%%%%%%%%%%
%%%%%%%%%%%%%%%%%%%%%%%%%%%%%%%%%%%%%%%%%%%%%%%%%%%%%%%%%%%%%%%%%%%%%%%%%%%%%%%
\begin{frame}[fragile]{\insertsubsection{}: Equazioni non in linea}
\begin{itemize}
\item Un'equazione lunga e complessa, andrebbe mostrata a parte usando
\cmdbegin{equation} e \cmdend{equation}.\\[2ex]
\begin{exampletwouptiny}
Le radici di un'equazione quadratica
sono date da
\begin{equation}
x = \frac{-b \pm \sqrt{b^2 - 4ac}}
         {2a}
\end{equation}
dove $a$, $b$ e $c$ sono \ldots
\end{exampletwouptiny}
\vskip 1em
{\scriptsize Attenzione: \LaTeX{} ignora gli spazi nella matematica, ma non
\`e in grado di gestire le line vuote --- non usatele!}
\end{itemize}
\end{frame}

%%%%%%%%%%%%%%%%%%%%%%%%%%%%%%%%%%%%%%%%%%%%%%%%%%%%%%%%%%%%%%%%%%%%%%%%%%%%%%%
%%%%%%%%%%%%%%%%%%%%%%%%%%%%%%%%%%%%%%%%%%%%%%%%%%%%%%%%%%%%%%%%%%%%%%%%%%%%%%%
%%%%%%%%%%%%%%%%%%%%%%%%%%%%%%%%%%%%%%%%%%%%%%%%%%%%%%%%%%%%%%%%%%%%%%%%%%%%%%%
\begin{frame}[fragile]{Intermezzo: Ambienti}
\begin{itemize}
\item \bftt{equation} \`e un \emph{ambiente} --- un contesto sematico.
\item Lo stesso comando pu\`o produrre risultati diversi in contesti diversi.
\begin{exampletwouptiny}
Possiamo scrivere
$ \Omega = \sum_{k=1}^{n} \omega_k $
in linea, o usare un ambiente
\begin{equation}
  \Omega = \sum_{k=1}^{n} \omega_k
\end{equation}
per mostrarlo.
\end{exampletwouptiny}
\vskip 2ex
\item Nota che $\Sigma$ \`e pi\`u grande nell'ambiente \bftt{equation}, e
che apici e pedici cambiano di posizione, nonostante si siano usati gli
stessi comandi
\vskip 1em
{\scriptsize Per inciso, avremmo potuto scrivere \bftt{\$...\$} come
\cmdbegin{math}\bftt{...}\cmdend{math}.}
\end{itemize}
\end{frame}

%%%%%%%%%%%%%%%%%%%%%%%%%%%%%%%%%%%%%%%%%%%%%%%%%%%%%%%%%%%%%%%%%%%%%%%%%%%%%%%
%%%%%%%%%%%%%%%%%%%%%%%%%%%%%%%%%%%%%%%%%%%%%%%%%%%%%%%%%%%%%%%%%%%%%%%%%%%%%%%
%%%%%%%%%%%%%%%%%%%%%%%%%%%%%%%%%%%%%%%%%%%%%%%%%%%%%%%%%%%%%%%%%%%%%%%%%%%%%%%
\begin{frame}[fragile]{Intermezzo: Ambienti}
\begin{itemize}
\item I comandi \cmdbs{begin} e \cmdbs{end} si possono usare per creare molti
ambienti diversi.
\vskip 2ex

\item Gli ambienti \bftt{itemize} ed \bftt{enumerate} generano liste.
\begin{exampletwouptiny}
\begin{itemize} % for bullet points
\item Biscotti
\item T\'e
\end{itemize}

\begin{enumerate} % for numbers
\item Biscotti
\item T\'e
\end{enumerate}
\end{exampletwouptiny}
\end{itemize}
\end{frame}

%%%%%%%%%%%%%%%%%%%%%%%%%%%%%%%%%%%%%%%%%%%%%%%%%%%%%%%%%%%%%%%%%%%%%%%%%%%%%%%
%%%%%%%%%%%%%%%%%%%%%%%%%%%%%%%%%%%%%%%%%%%%%%%%%%%%%%%%%%%%%%%%%%%%%%%%%%%%%%%
%%%%%%%%%%%%%%%%%%%%%%%%%%%%%%%%%%%%%%%%%%%%%%%%%%%%%%%%%%%%%%%%%%%%%%%%%%%%%%%
\begin{frame}[fragile]{Intermezzo: Pacchetti}

\begin{itemize}
\item Tutti i comandi e gli ambienti mostrati fino ad adesso, sono parte di
\LaTeX.

\item I \emph{pacchetti} sono librerie di comandi e ambienti aggiuntivi.
Ci sono migliaia di pacchetti liberamente disponibili.

\item I pacchetti che vogliamo usare vanno caricati esplicitamente usando il comando
\cmdbs{usepackage} nel \emph{preambolo}.

\item Esempio: \bftt{amsmath} della American Mathematical Society.
\begin{minted}[fontsize=\small,frame=single]{latex}
\documentclass{article}
\usepackage{amsmath} % preamble
\begin{document}
% now we can use commands from amsmath here...
\end{document}
\end{minted}
\end{itemize}
\end{frame}

%%%%%%%%%%%%%%%%%%%%%%%%%%%%%%%%%%%%%%%%%%%%%%%%%%%%%%%%%%%%%%%%%%%%%%%%%%%%%%%
%%%%%%%%%%%%%%%%%%%%%%%%%%%%%%%%%%%%%%%%%%%%%%%%%%%%%%%%%%%%%%%%%%%%%%%%%%%%%%%
%%%%%%%%%%%%%%%%%%%%%%%%%%%%%%%%%%%%%%%%%%%%%%%%%%%%%%%%%%%%%%%%%%%%%%%%%%%%%%%
\begin{frame}[fragile]{\insertsubsection{}: Esempi con \bftt{amsmath}}
\begin{itemize}
\item Usa \bftt{equation*} per inserire equazioni non numerate.
\begin{exampletwouptiny}
\begin{equation*}
  \Omega = \sum_{k=1}^{n} \omega_k
\end{equation*}
\end{exampletwouptiny}
\item \LaTeX{} tratta lettere adiacenti come variabili moltiplicate l'una con l'altra, cosa che non \`e sempre desiderabile. \bftt{amsmath} definisce comandi per la maggior parte delle funzioni matematiche.
\begin{exampletwouptiny}
\begin{equation*} % sbagliato!
 min_{x,y} (1-x)^2 + 100(y-x^2)^2
\end{equation*}
\begin{equation*} % giusto!
\min_{x,y}{(1-x)^2 + 100(y-x^2)^2}
\end{equation*}
\end{exampletwouptiny}
\item Per quelle non predefinite, si usa \cmdbs{operatorname}.
\begin{exampletwouptiny}
\begin{equation*}
\beta_i =
\frac{\operatorname{Cov}(R_i, R_m)}
     {\operatorname{Var}(R_m)}
\end{equation*}
\end{exampletwouptiny}
\end{itemize}
\end{frame}

%%%%%%%%%%%%%%%%%%%%%%%%%%%%%%%%%%%%%%%%%%%%%%%%%%%%%%%%%%%%%%%%%%%%%%%%%%%%%%%
%%%%%%%%%%%%%%%%%%%%%%%%%%%%%%%%%%%%%%%%%%%%%%%%%%%%%%%%%%%%%%%%%%%%%%%%%%%%%%%
%%%%%%%%%%%%%%%%%%%%%%%%%%%%%%%%%%%%%%%%%%%%%%%%%%%%%%%%%%%%%%%%%%%%%%%%%%%%%%%
\begin{frame}[fragile]{\insertsubsection{}: Esempi con \bftt{amsmath}}
\begin{itemize}{\small
\item Allinea una sequenza di equazioni con il simbolo di uguale
\begin{align*}
(x+1)^3 &= (x+1)(x+1)(x+1) \\
        &= (x+1)(x^2 + 2x + 1) \\
        &= x^3 + 3x^2 + 3x + 1
\end{align*}
utilizzando l'ambiente \bftt{align*}.

% for whatever reason, this doesn't play well with the twoup environment
\begin{minted}[fontsize=\small,frame=single]{latex}
\begin{align*}
(x+1)^3 &= (x+1)(x+1)(x+1) \\
        &= (x+1)(x^2 + 2x + 1) \\
        &= x^3 + 3x^2 + 3x + 1
\end{align*}
\end{minted}
\item Una ampersand \keystrokebftt{\&} separa la colonna sinistra (prima di
$=$) dalla colonna destra (dopo di $=$).
\item Un doppio backslash \keystrokebftt{\bs}\keystrokebftt{\bs} inizia una
nuova linea.
}\end{itemize}
\end{frame}


%%%%%%%%%%%%%%%%%%%%%%%%%%%%%%%%%%%%%%%%%%%%%%%%%%%%%%%%%%%%%%%%%%%%%%%%%%%%%%%
%%%%%%%%%%%%%%%%%%%%%%%%%%%%%%%%%%%%%%%%%%%%%%%%%%%%%%%%%%%%%%%%%%%%%%%%%%%%%%%
%%%%%%%%%%%%%%%%%%%%%%%%%%%%%%%%%%%%%%%%%%%%%%%%%%%%%%%%%%%%%%%%%%%%%%%%%%%%%%%
\begin{frame}[fragile]{Esercizio 2}

\begin{block}{Digita questo in \LaTeX:}
Sia $X_1, X_2, \ldots, X_n$ una serie di variabili casuali indipendenti ed
identicamente distribuite con media $\operatorname{E}[X_i] = \mu$ e varianza
$\operatorname{Var}[X_i] = \sigma^2 < \infty$, laddove
\begin{equation*}
S_n = \frac{1}{n}\sum_{i}^{n} X_i
\end{equation*}
denoti la loro media. Per $n$ che tende ad infinito, le variabili casuali
$\sqrt{n}(S_n - \mu)$ convergono in senso distribuzionale ad una gaussiana $N(0, \sigma^2)$.
\end{block}
\vskip 2ex
\begin{center}
\fbox{\href{\wlnewdoc{basics-exercise-2.tex}}{%
Clicca per aprire questo esercizio in \wllogo{}}}
\end{center}
\begin{itemize}
\item Suggerimento: il comando per $\infty$ \`e \cmdbs{infty}.
\item Qui puoi trovare la 
\fbox{\href{\wlnewdoc{basics-exercise-2-solution.tex}}{%
mia soluzione}}.
\end{itemize}
\end{frame}

%%%%%%%%%%%%%%%%%%%%%%%%%%%%%%%%%%%%%%%%%%%%%%%%%%%%%%%%%%%%%%%%%%%%%%%%%%%%%%%
%%%%%%%%%%%%%%%%%%%%%%%%%%%%%%%%%%%%%%%%%%%%%%%%%%%%%%%%%%%%%%%%%%%%%%%%%%%%%%%
%%%%%%%%%%%%%%%%%%%%%%%%%%%%%%%%%%%%%%%%%%%%%%%%%%%%%%%%%%%%%%%%%%%%%%%%%%%%%%%
\begin{frame}{Fine della prima parte}
\begin{itemize}
\item Congratulazioni! Avete imparato a\ldots
\begin{itemize}
\item Digitare testo in \LaTeX.
\item Utilizzare molti dei comandi di base.
\item Gestire gli errori via via che compaiono.
\item Scrivere belissima matematica.
\item Utilizzare alcuni degli ambienti.
\item Caricare pacchetti.
\end{itemize}
\item Non \`e fantastico?
\item Nella seconda parte, impareremo ad usare \LaTeX{} per scrivere documenti
strutturati con sezioni, riferimenti incrociati, figure, tabelle e bigliografia.
Alla prossima!
\end{itemize}
\end{frame}

\end{document}
