\documentclass{article}
\usepackage{amsmath}
\begin{document}

Siano $X_1, X_2, \ldots, X_n$ una serie di variabili casuali indipendenti ed
identicamente distribuite con media $\operatorname{E}[X_i] = \mu$ e varianza
$\operatorname{Var}[X_i] = \sigma^2 < \infty$, con media
\begin{equation*}
S_n = \frac{1}{n}\sum_{i}^{n} X_i
\end{equation*}
Per $n$ che tende ad infinito, le variabili casuali
$\sqrt{n}(S_n - \mu)$ convergono in senso distribuzionale
ad una gaussiana $N(0, \sigma^2)$.

% punti bonus: solitamente la N della distribuzione normale
% viene scritta con un font calligrafico.
% per usarlo: $\mathcal{N}(0, \sigma^2)$.

\end{document}

